
%%%%%%%%%%%%%%%%%%%%%%%%%%%%%%%%%%%%%%%%%per fare le conclusioni
\chapter*{Conclusioni}
%%%%%%%%%%%%%%%%%%%%%%%%%%%%%%%%%%%%%%%%%imposta l'intestazione di pagina
\rhead[\fancyplain{}{\bfseries
Conclusioni}]{\fancyplain{}{\bfseries\thepage}}
\lhead[\fancyplain{}{\bfseries\thepage}]{\fancyplain{}{\bfseries
CONCLUSIONI}}
%%%%%%%%%%%%%%%%%%%%%%%%%%%%%%%%%%%%%%%%%aggiunge la voce Conclusioni
                                        %   nell'indice
\addcontentsline{toc}{chapter}{Conclusioni}
Lo state dell'arte al momento in cui si scrive \`e una libreria in grado di configurare uno stack per LWIPv6 completamente, mentre gli altri moduli sono in fase di sviluppo ma esistono gi\`a dei proof of concept (moduli in fase sperimentale) all'interno del repo su github\footnote{http://github.com/simonepreite/vnslib}.
I risultati dei test effettuati, nonostante il progetto sia tutt'ora in fase di sviluppo, sono significativi: si riesce a configurare completamente uno stack specifico in maniera generica ed usando le interfacce proposte dal kernel Linux.\\
L'utilizzo di queste tipologie di stack dipende strettamente dalla diffusione del protocollo IPv6 e dei sistemi embedded.\\
Lo sviluppo di queste tecnologie \`e punti chiave per il testing del procollo IPv6. Solo mantenendo costante la ricerca si pu\`o pensare di creare sistemi stabili quando IPv6 sar\`a
La ricerca e lo sviluppo di questi meccanismi sono necessari per ottenere un sistema performante e stabile quando IPv6 sar\`a l'ordinario. Vanno inoltre considerati i vantaggi conseguenti nelle reti virtuali che gi\`a ora vengono utilizzate per la sperimentazione e che in futuro saranno sicuramente di larga diffusione.\\
L'indipendenza dalle infrastrutture fisiche \`e un vantaggio non indifferente, significa che le reti ed i sistemi possono essere creati, modificati o ricostruiti da zero con estrema facilit\`a. Basti pensare ad una rete di macchine virtuali che, trattandosi di software, pu\`o essere ampliata con nuovi elaboratori virtuali, spenta e modificata senza dover invesitire in nuovo hardware.\\
Quelli esposti sono solo alcuni dei vantaggi sufficienti a lasciare intendere la direzione verso cui questo settore sta proseguendo.
