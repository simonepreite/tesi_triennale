

\chapter{VsnLib}                %crea il capitolo
%%%%%%%%%%%%%%%%%%%%%%%%%%%%%%%%%%%%%%%%%imposta l'intestazione di pagina
\lhead[\fancyplain{}{\bfseries\thepage}]{\fancyplain{}{\bfseries\rightmark}}
Strato di compatibilit\`a tra pacchetti netlink e stack di rete virtuali.

\section{Il Progetto}                 %crea la sezione
Gli stack analizzati in precedenza offrono a grandi linee la stessa tipologia di servizio seppur ognuna con le proprie caratteristice.\\
Nessuno dei progetti ha per\`o tenuto in considerazione l'idea di utilizzare un'interfaccia di configurazione che fosse standard e pertanto \`e necessario usare le funzioni specifiche per ognuno di questi progetti, costringendo il programmatore a cambiare modalit\`a di interazione da stack a stack.\\
Il progetto di creare questa libreria nasce proprio da questa esigenza, ovvero cercare di uniformare le interfacce di comunicazione in modo tale che l'utente non sia costretto ad adattarsi ogni qual volta decida di cambiare stack.\\
\section{VnsLib}
\subsection{Preambolo}
Per configurare uno stack di rete programmi come ip generano un pacchetto netlink con le informazioni necessarie lo inviano al kernel che lo elabora esegue le operazioni e genera un messaggio di errore nel quale esiste un flag contenente il numero di errore generato (0 in caso di successo), a questo messaggio \`e associato un payload di risposta che pu\`o essere anche un feedback per l'utente che sta interagendo con il programma.\\

\subsection{Ambiente}
La necessit\`a primaria era quella di catturare questi netlink e per farlo la strada era quella di intercettare la system call di invio della richiesta contenente appunto il payload, questo momento \`e quello in cui viene eseguita la sendto.\\
In questo punto interviene la nostra libreria, che si interpone e di fatto fa le veci del kernel, al quale il pacchetto netlink non arriver\`a mai realmente.\\
Il progetto inizialmente utilizzava come motore di cattura delle system call la libreria purelibc\footnote{http://wiki.virtualsquare.org/wiki/index.php/PureLibc}, in seguito per\`o si \`e giunti alla conclusione che costruire un modulo ad hoc all'interno del sistema vuos fosse un'alternativa pi\`u versatile, immediata e pulita, inoltre vuos ha un sistema di debug built in che rende pi\`u semplice l'individuazione dei problemi legati allo sviluppo.\\
Il modulo in questione si chiama unrealvsnlib per analogia alla libreria ma ognuno potrebbe costruire un suo modulo a seconda di quello che intende controllare.\\

\subsection{Core}
La libreria risulta come uno strato di compatibilit\`a tra netlink e le specifiche configurazioni degli stack.\\
Vediamo come \`e composta:
\begin{description}                     %crea un elenco descrittivo
  \item[VnsLib:] Il primo stato si occupa di inizializzare la libreria in base allo stack che si intende utilizzare, in questo modo possiamo caricare dinamicamente solo il modulo che ci interessa.
  \item[Modules:] Sono la parte specifica della libreria, essi contengono l'intestazione delle funzioni generiche che si occupano di chiamare quelle particolari per ogni stack.
\end{description}
\begin{figure}[h]                       %crea l'ambiente figura; [h] sta
                                        %   per here, cio� la figura va qui
\begin{center}                          %centra nel mezzo della pagina
                                        %   la figura
\includegraphics[width=15cm]{vsnlib_scheme}%inserisce una figura larga 5cm
                                        %se si vuole usare va scommentata
%
%%%%%%%%%%%%%%%%%%%%%%%%%%%%%%%%%%%%%%%%%inserisce la legenda ed etichetta
                                        %   la figura con \label{fig:prima}
\caption[mappa concettuale libreria]{mappa concettuale libreria}\label{fig:map}
\end{center}
\end{figure}



\subsection{Moduli}
La potenza dei moduli risiede nel fatto che chiunque abbia intenzione di costruire uno stack personalizzato, o di utilizzarne uno per il quale non esiste un modulo, pu\`o facilmente realizzare il proprio e la libreria si occuper\`a di farne il caricamento qualora fosse richiesto.
\section{Sviluppi Futuri}
%%%%%%%%%%%%%%%%%%%%%%%%%%%%%%%%%%%%%%%%%non numera l'ultima pagina sinistra
\clearpage{\pagestyle{empty}\cleardoublepage} 
