

\chapter*{Introduzione}                 %crea l'introduzione (un capitolo
                                        %   non numerato)

%%%%%%%%%%%%%%%%%%%%%%%%%%%%%%%%%%%%%%%%%imposta l'intestazione di pagina
\rhead[\fancyplain{}{\bfseries
INTRODUZIONE}]{\fancyplain{}{\bfseries\thepage}}
\lhead[\fancyplain{}{\bfseries\thepage}]{\fancyplain{}{\bfseries
INTRODUZIONE}}
%%%%%%%%%%%%%%%%%%%%%%%%%%%%%%%%%%%%%%%%%aggiunge la voce Introduzione
                                        %   nell'indice
\addcontentsline{toc}{chapter}{Introduzione}

Non vi \`e dubbio che i nodi nella rete internet siano cresciuti in maniera esponenziale fino ad oggi. Nonostante l'enorme dimensione che ha raggiunto oggi questa rete ha mantenuto gli stessi principi costruttivi che impiegava quando collegava poche macchine sperimentali: I nodi della rete sono considerati ancora le interfacce di rete degli elaboratori. Questa situazione non \`e pi\`u coerente con il funzionamento della rete interent odierna, dove il ruolo fondamentale non \`e pi\`u quello delle macchine che erogano servizi bens\`i ai servizi stessi ed ai processi li forniscono.\\
La crescita smisurata dei nodi di rete e la necessit\`a di passare da un'indirizzamento di nodi all'indirizzamento di processi genera il bisogno di avere una grande quantit\`a di indirizzi per poter accedere questi processi individualmente. La famiglia di protoccoli, oggi pi\`u in uso, IP vesione 4 (IPv4) risulta insufficiente per questo scopo, infatti vengono usati solamente 4 byte per indirizzare tutti i nodi accedibili direttamente.\\
Per questo gi\`a da tempo \`e in atto la migrazione verso il protocollo versione 6 del procollo IP (IPv6) che consente di avere una quantit\`a di indirizzi molto maggiore, sia per la problematica che abbiamo visto prima sia perch\`e sempre pi\`u si affacciano sulla rete dispositivi automatici capaci di agire come entit\`a autonome, oggetti provvisti di collegamenti di rete; \`e il fenomeno del cos\`i detto internet of thing o IoT.\\
In questa ottica si colloca il presente lavoro di tesi. La realizzazione di strumenti dell'internet of threads (IoTh), cio\`e la possibilit\`a di assegnare indirizzi anche a singoli processi e thread, richiede l'utilizzo di stack di rete a livello del singolo processo ovvero implementati come librerie e funzionanti in spazio utente.\\
Se l'interfaccia di uso comune da parte delle applicazioni della rete ha come standard la api di Berkeley socket non esiste altrettanto standardizzata interfaccia per la configurazione e la gestione delle interfacce di rete, essendo queste, nell'ottica degli elaboratori come nodi di rete, prerogative riservate agli amministratori di sistema e non comunemente accessibili come api da parte di processi.\\
In questa tesi si studier\`a come realizzare una libreria che sia in grado di interfacciarsi ad applicazioni che usano il protocollo netlink, protocollo usato dal kernel linux per la configurazione e la gestione delle interfacce di rete, in modo da poter utilizzare, all'interno dei programmi, gli strumenti per gli amministratori di rete.\\
Nella parte seguente dell'introduzione si illustrer\`a la struttura della seguente tesi.\\
Il Prossimo capitolo presenta una rassegna dello stato dell'arte nell'ambito dei protocolli di rete e del paradigma di IoTh; nel secondo capitolo verr\`a trattato il progetto VsnLib, ovvero la libreria oggetto della tesi; infine il terzo capitolo descriver\`a alcuni esempi di utilizzo della stessa.
%%%%%%%%%%%%%%%%%%%%%%%%%%%%%%%%%%%%%%%%%non numera l'ultima pagina sinistra
\clearpage{\pagestyle{empty}\cleardoublepage}
\tableofcontents                        %crea l'indice
%%%%%%%%%%%%%%%%%%%%%%%%%%%%%%%%%%%%%%%%%imposta l'intestazione di pagina
\rhead[\fancyplain{}{\bfseries\leftmark}]{\fancyplain{}{\bfseries\thepage}}
\lhead[\fancyplain{}{\bfseries\thepage}]{\fancyplain{}{\bfseries
INDICE}}
%%%%%%%%%%%%%%%%%%%%%%%%%%%%%%%%%%%%%%%%%non numera l'ultima pagina sinistra
\clearpage{\pagestyle{empty}\cleardoublepage}
\listoffigures                          %crea l'elenco delle figure
%%%%%%%%%%%%%%%%%%%%%%%%%%%%%%%%%%%%%%%%%non numera l'ultima pagina sinistra
\clearpage{\pagestyle{empty}\cleardoublepage}
\listoftables                           %crea l'elenco delle tabelle
%%%%%%%%%%%%%%%%%%%%%%%%%%%%%%%%%%%%%%%%%non numera l'ultima pagina sinistra
\clearpage{\pagestyle{empty}\cleardoublepage}
