

\chapter*{Introduzione}                 %crea l'introduzione (un capitolo
                                        %   non numerato)

%%%%%%%%%%%%%%%%%%%%%%%%%%%%%%%%%%%%%%%%%imposta l'intestazione di pagina
\rhead[\fancyplain{}{\bfseries
INTRODUZIONE}]{\fancyplain{}{\bfseries\thepage}}
\lhead[\fancyplain{}{\bfseries\thepage}]{\fancyplain{}{\bfseries
INTRODUZIONE}}
%%%%%%%%%%%%%%%%%%%%%%%%%%%%%%%%%%%%%%%%%aggiunge la voce Introduzione
                                        %   nell'indice
\addcontentsline{toc}{chapter}{Introduzione}

Nell'ambito delle reti negli ultimi anni con l'avvento di internet si \`e assistito ad una crescita esponenziale delle apparecchiature connesse a questa rete.
Fino ad oggi questa immensa rete \`e stata gestita per lo pi\`u attraverso indirizzamento delle interfacce di rete di cui i computer sono dotati.\\
Al momento per\`o ci si trova a fare i conti con la carenza degli indirizzi e pian piano si procede verso una migrazione basata sul nuovo (ma non troppo) protocollo IPv6 che prevede uno spazio di indirizzamento molto pi\`u ampio e lascia spazio a nuove idee impraticabili con protocollo IPv4.\\
In seguito sentiremo parlare di IoT (Internet of Things) ed IoTh (Internet of Threads), due paradigmi che possiamo affermare essere il risultato di un'evoluzione sempre crescente dei sistemi informativi.\\
L'elaborato sar\`a strutturato in capitoli.\\
Il primo di questi si premura di introdurre il lettore allo stato attuale dei protolli di rete e del paradigma IoTh; nel secondo capitolo verr\`a trattato il progetto VsnLib, ovvero la libreria oggetto della tesi; in fine il terzo capito si occuper\`a di descrivere alcuni esempi di utilizzo della stessa con immagini dimostrative ed istruzioni.
%%%%%%%%%%%%%%%%%%%%%%%%%%%%%%%%%%%%%%%%%non numera l'ultima pagina sinistra
\clearpage{\pagestyle{empty}\cleardoublepage}
\tableofcontents                        %crea l'indice
%%%%%%%%%%%%%%%%%%%%%%%%%%%%%%%%%%%%%%%%%imposta l'intestazione di pagina
\rhead[\fancyplain{}{\bfseries\leftmark}]{\fancyplain{}{\bfseries\thepage}}
\lhead[\fancyplain{}{\bfseries\thepage}]{\fancyplain{}{\bfseries
INDICE}}
%%%%%%%%%%%%%%%%%%%%%%%%%%%%%%%%%%%%%%%%%non numera l'ultima pagina sinistra
\clearpage{\pagestyle{empty}\cleardoublepage}
\listoffigures                          %crea l'elenco delle figure
%%%%%%%%%%%%%%%%%%%%%%%%%%%%%%%%%%%%%%%%%non numera l'ultima pagina sinistra
\clearpage{\pagestyle{empty}\cleardoublepage}
\listoftables                           %crea l'elenco delle tabelle
%%%%%%%%%%%%%%%%%%%%%%%%%%%%%%%%%%%%%%%%%non numera l'ultima pagina sinistra
\clearpage{\pagestyle{empty}\cleardoublepage} 

