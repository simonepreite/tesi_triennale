

%%%%%%%%%%%%%%%%%%%%%%%%%%%%%%%%%%%%%%%%%12pt: grandezza carattere
                                        %a4paper: formato a4
                                        %openright: apre i capitoli a destra
                                        %twoside: serve per fare un
                                        %   documento fronteretro
                                        %report: stile tesi (oppure book)
\documentclass[14pt,a4paper,openright,twoside]{extreport}
%
%%%%%%%%%%%%%%%%%%%%%%%%%%%%%%%%%%%%%%%%%libreria per scrivere in italiano
\usepackage[italian]{babel}
%
%%%%%%%%%%%%%%%%%%%%%%%%%%%%%%%%%%%%%%%%%libreria per accettare i caratteri
                                        %   digitati da tastiera come � �
                                        %   si pu� usare anche
                                        %   \usepackage[T1]{fontenc}
                                        %   per� con questa libreria
                                        %   il tempo di compilazione
                                        %   aumenta
\usepackage{newlfont}
\textwidth=450pt\oddsidemargin=0pt
                        
                                        
\usepackage[latin1]{inputenc}
%
%%%%%%%%%%%%%%%%%%%%%%%%%%%%%%%%%%%%%%%%%libreria per impostare il documento
\usepackage{fancyhdr}
%
%%%%%%%%%%%%%%%%%%%%%%%%%%%%%%%%%%%%%%%%%libreria per avere l'indentazione
%%%%%%%%%%%%%%%%%%%%%%%%%%%%%%%%%%%%%%%%%   all'inizio dei capitoli, ...
\usepackage{indentfirst}
%
%%%%%%%%%libreria per mostrare le etichette
%\usepackage{showkeys}
%
%%%%%%%%%%%%%%%%%%%%%%%%%%%%%%%%%%%%%%%%%libreria per inserire grafici
\usepackage{graphicx}
%
%%%%%%%%%%%%%%%%%%%%%%%%%%%%%%%%%%%%%%%%%libreria per utilizzare font
                                        %   particolari ad esempio
                                        %   \textsc{}
\usepackage{newlfont}
%
%%%%%%%%%%%%%%%%%%%%%%%%%%%%%%%%%%%%%%%%%librerie matematiche
\usepackage{amssymb}
\usepackage{amsmath}
\usepackage{latexsym}
\usepackage{amsthm}
%
\oddsidemargin=10pt \evensidemargin=20pt%impostano i margini
\hyphenation{sil-la-ba-zio-ne pa-ren-te-si}%serve per la sillabazione: tra parentesi 
					   %vanno inserite come nell'esempio le parole 
%					   %che latex non riesce a tagliare nel modo giusto andando a capo.

%
%%%%%%%%%%%%%%%%%%%%%%%%%%%%%%%%%%%%%%%%%comandi per l'impostazione
                                        %   della pagina, vedi il manuale
                                        %   della libreria fancyhdr
                                        %   per ulteriori delucidazioni
\pagestyle{fancy}\addtolength{\headwidth}{20pt}
\renewcommand{\chaptermark}[1]{\markboth{\thechapter.\ #1}{}}
\renewcommand{\sectionmark}[1]{\markright{\thesection \ #1}{}}
\rhead[\fancyplain{}{\bfseries\leftmark}]{\fancyplain{}{\bfseries\thepage}}
\cfoot{}
%%%%%%%%%%%%%%%%%%%%%%%%%%%%%%%%%%%%%%%%%
\linespread{1.3}                        %comando per impostare l'interlinea
%%%%%%%%%%%%%%%%%%%%%%%%%%%%%%%%%%%%%%%%%definisce nuovi comandi
%
\begin{document}

\begin{titlepage}
\begin{center}
{{\Large{\textsc{Alma Mater Studiorum $\cdot$ Universit\`a di
Bologna}}}} \rule[0.1cm]{15.8cm}{0.1mm}
\rule[0.5cm]{15.8cm}{0.6mm}
{\small{\bf SCUOLA DI SCIENZE\\
Corso di Laurea in Informatica }}
\end{center}
\vspace{15mm}
\begin{center}
{\LARGE{\bf VnsLib:}}\\
\vspace{3mm}
{\LARGE{\bf una interfaccia unica di configurazione}}\\
{\LARGE{\bf per stack di rete virtuali tramite pacchetti netlink}}\\
\end{center}
\vspace{40mm}
\par
\noindent
\begin{minipage}[t]{0.47\textwidth}
{\large{\bf Relatore:\\
Chiar.mo Prof.\\
Renzo Davoli}}
\end{minipage}
\hfill
\begin{minipage}[t]{0.47\textwidth}\raggedleft
{\large{\bf Presentata da:\\
Simone Preite}}
\end{minipage}
\vspace{20mm}
\begin{center}
{\large{\bf Sessione di Dicembre\\%inserire il numero della sessione in cui ci si laurea
Anno Accademico 2016-2017}}%inserire l'anno accademico a cui si � iscritti
\end{center}
\end{titlepage}







\begin{titlepage}                       %crea un ambiente libero da vincoli
                                        %   di margini e grandezza caratteri:
                                        %   si pu\`o modificare quello che si
                                        %   vuole, tanto fuori da questo
                                        %   ambiente tutto viene ristabilito
%
\thispagestyle{empty}                   %elimina il numero della pagina
\topmargin=6.5cm                        %imposta il margina superiore a 6.5cm
\raggedleft                             %incolonna la scrittura a destra
\large                                  %aumenta la grandezza del carattere
                                        %   a 14pt
\em                                     %emfatizza (corsivo) il carattere
Questa \`e la \textsc{Dedica}:\\
ognuno pu\`o scrivere quello che vuole, \\
anche nulla \ldots                      %\ldots lascia tre puntini
\newpage                                %va in una pagina nuova
%
%%%%%%%%%%%%%%%%%%%%%%%%%%%%%%%%%%%%%%%%
\clearpage{\pagestyle{empty}\cleardoublepage}%non numera l'ultima pagina sinistra
\end{titlepage}
\pagenumbering{roman}                   %serve per mettere i numeri romani
\chapter*{Introduzione}                 %crea l'introduzione (un capitolo
                                        %   non numerato)
%%%%%%%%%%%%%%%%%%%%%%%%%%%%%%%%%%%%%%%%%imposta l'intestazione di pagina
\rhead[\fancyplain{}{\bfseries
INTRODUZIONE}]{\fancyplain{}{\bfseries\thepage}}
\lhead[\fancyplain{}{\bfseries\thepage}]{\fancyplain{}{\bfseries
INTRODUZIONE}}
%%%%%%%%%%%%%%%%%%%%%%%%%%%%%%%%%%%%%%%%%aggiunge la voce Introduzione
                                        %   nell'indice
\addcontentsline{toc}{chapter}{Introduzione}
IoTh � un concetto che nasce dall'esigenza di astrarre una tecnologia obsoleta che � quella di voler interconnettere computer, o meglio interfacce di rete.\\
L'evoluzione delle reti e l'avvento di IPv6 hanno permettono una migrazione verso un paradigma completamente ripensato che per� si integra in maniera trasparente con il sistema attualmente in uso.\\
L'idea \`e quindi quella di permettere ai processi di avere un proprio indirizzo ed essere quindi veri e propri endpoint su internet, tra le soluzioni \`e emersa quella di associare ad ogni processo il proprio stack di rete.\\
Diversi sono i vantaggi derivanti dalla sopraesposta soluzione, per esempio la migrazione di un servizio su un altro host equivale ad interromperne l'esecuzione sulla macchina che lo ospita e rieseguirlo sulla nuova; questo manterebbe il suo indirizzo e quindi non si presenta la necessit� di dover comunicare a chiunque lo utilizzi la variazione di indirizzo.
Un altro vantaggio \`e quello di non vicolnare il programmatore allo stack di rete di sistema, infatti esso pu\`o scegliere di usare, tra gli stack disponibili o creandone uno ex-novo, quello che pi\`u si addice alle sue esigenze.\\
Per quanto riguarda la sicurezza invece il nuovo paradigma \`e intrinsecamente solido, avendo i processi il proprio stack di rete e quindi il proprio indirizzo i software di port mapping non riuscirebbero a carpire informazioni  sensibili sui servizi attivi in quanto riceverebbero solo la porta di default per quel servizio specifico senza considerare che quel servizio specifico avendo il proprio stack possono essere eseguti come utente non privilegiato e pertanto una vulnerabilit\`a di quest'ultimo non comprometterebbe l'intero sistema.\\
Esistono gi\`a differenti implentazioni di stack di rete di questo genere, tra i pi\`u famosi abbiamo picoTCP, lwip e lwipv6; quest'ultimo nasce dal lavoro del team di Virtual Square labs come fork di lwip con l'obiettivo di fornirvi supporto ipv6. Con il tempo poi \`e diventato un progetto parallelo a se stante.\\
Il problema principale degli stack di rete di questo genere \`e che ognuno ha disegnato la propria interfaccia di configurazione, pertanto gli utilizzatori devono studiare ed abituarsi a configurare ogni stack che utilizzano.\\
Considerando che la configurazione dello stack di rete tendenzialemente richiede gli stessi parametri risulta scomodo il fatto di dover utilizzare motodi diversi su stack diversi, da questa problematica si \`e sentito il bisogno di unificare l'interfaccia di configurazione con una unica che riesca ad essere abbastanza generale e semplice.\\
Rispondiamo a questa esigenza attraverso la creazione di una libreria che possa catturare i pacchetti netlink e ne utilizzi le informazioni per poi chiamare, in base allo stack di rete in uso, le funzioni adeguate a configurarlo.\\
In questo modo si permette allo sviluppatore di utilizzare le interfacce con cui configura normalmente lo stack di rete del kernel.
%%%%%%%%%%%%%%%%%%%%%%%%%%%%%%%%%%%%%%%%%non numera l'ultima pagina sinistra
\clearpage{\pagestyle{empty}\cleardoublepage}
\tableofcontents                        %crea l'indice
%%%%%%%%%%%%%%%%%%%%%%%%%%%%%%%%%%%%%%%%%imposta l'intestazione di pagina
\rhead[\fancyplain{}{\bfseries\leftmark}]{\fancyplain{}{\bfseries\thepage}}
\lhead[\fancyplain{}{\bfseries\thepage}]{\fancyplain{}{\bfseries
INDICE}}
%%%%%%%%%%%%%%%%%%%%%%%%%%%%%%%%%%%%%%%%%non numera l'ultima pagina sinistra
\clearpage{\pagestyle{empty}\cleardoublepage}
\listoffigures                          %crea l'elenco delle figure
%%%%%%%%%%%%%%%%%%%%%%%%%%%%%%%%%%%%%%%%%non numera l'ultima pagina sinistra
\clearpage{\pagestyle{empty}\cleardoublepage}
\listoftables                           %crea l'elenco delle tabelle
%%%%%%%%%%%%%%%%%%%%%%%%%%%%%%%%%%%%%%%%%non numera l'ultima pagina sinistra
\clearpage{\pagestyle{empty}\cleardoublepage}
\chapter{Internet of Threads}                %crea il capitolo
%%%%%%%%%%%%%%%%%%%%%%%%%%%%%%%%%%%%%%%%%imposta l'intestazione di pagina
\lhead[\fancyplain{}{\bfseries\thepage}]{\fancyplain{}{\bfseries\rightmark}}
\pagenumbering{arabic}                  %mette i numeri arabi
Questo \`e il primo capitolo.
\section{paradigma}                 %crea la sezione
Questa \`e la prima sezione.

Ora vediamo un elenco numerato:         %crea un elenco numerato
\begin{enumerate}
\item primo oggetto
\item secondo oggetto
\item terzo oggetto
\item quarto oggetto
\end{enumerate}

\begin{figure}[h]                       %crea l'ambiente figura; [h] sta
                                        %   per here, cio� la figura va qui
\begin{center}                          %centra nel mezzo della pagina
                                        %   la figura
%\includegraphics[width=5cm]{figura.eps}%inserisce una figura larga 5cm
                                        %se si vuole usare va scommentata
%
%%%%%%%%%%%%%%%%%%%%%%%%%%%%%%%%%%%%%%%%%inserisce la legenda ed etichetta
                                        %   la figura con \label{fig:prima}
\caption[legenda elenco figure]{legenda sotto la figura}\label{fig:prima}
\end{center}
\end{figure}

\section{Seconda Sezione}
Ora vediamo un elenco puntato:
\begin{itemize}                         %crea un elenco puntato
\item primo oggetto
\item secondo oggetto
\end{itemize}

\section{Altra Sezione}
Vediamo un elenco descrittivo:
\begin{description}                     %crea un elenco descrittivo
  \item[OGGETTO1] prima descrizione;
  \item[OGGETTO2] seconda descrizione;
  \item[OGGETTO3] terza descrizione.
\end{description}
%%%%%%%%%%%%%%%%%%%%%%%%%%%%%%%%%%%%%%%%%crea una sottosezione
\subsection{Altra SottoSezione}
%%%%%%%%%%%%%%%%%%%%%%%%%%%%%%%%%%%%%%%%%crea una sottosottosezione
\subsubsection{SottoSottoSezione}Questa sottosottosezione non viene
numerata, ma \`e solo scritta in grassetto.
\section{Altra Sezione}                 %crea una sottosezione
Vediamo la creaz-italian non precedentemente selezionato.
(Lettura del database... 233368 file ione di una tabella; la tabella \ref{tab:uno}
(richiamo il nome della tabella utilizzando la label che ho messo sotto):
la facciamo di tre righe e tre colonne, la prima colonna
``incolonnata'' a destra (r) e le altre centrate (c):\\
\begin{table}[h]                        %ambiente tabella
                                        %(serve per avere la legenda)
\begin{center}                          %centra nella pagina la tabella
\begin{tabular}{r|c|c}                  %tre colonne con righe verticali
                                        %   prodotte con |
\hline \hline                           %inserisce due righe orizzontali
$(1,1)$ & $(1,2)$ & $(1,3)$\\           %& separa le colonne e con
\hline                                  %inserisce una riga orizzontale
$(2,1)$ & $(2,2)$ & $(2,3)$\\           %  \\ va a capo
\hline                                  %inserisce una riga orizzontale
$(3,1)$ & $(3,2)$ & $(3,3)$\\
\hline \hline                           %inserisce due righe orizzontali
\end{tabular}
\caption[legenda elenco tabelle]{legenda tabella}\label{tab:uno}
\end{center}
\end{table}
\section{Altra Sezione}\label{sec:prova}%posso mettere le label anche
                                        %   alle section
\subsection{Listati dei programmi}
\subsubsection{Primo Listato}
\begin{verbatim}
        In questo ambiente     posso scrivere      come voglio,
lasciare gli spazi che voglio e non % commentare quando voglio
e ci sar� scritto tutto.
Quando lo uso � meglio che disattivi il Wrap del WinEdt
\end{verbatim}
%%%%%%%%%%%%%%%%%%%%%%%%%%%%%%%%%%%%%%%%%non numera l'ultima pagina sinistra

\clearpage{\pagestyle{empty}\cleardoublepage}

\chapter{VsnLib}                %crea il capitolo
%%%%%%%%%%%%%%%%%%%%%%%%%%%%%%%%%%%%%%%%%imposta l'intestazione di pagina
\lhead[\fancyplain{}{\bfseries\thepage}]{\fancyplain{}{\bfseries\rightmark}}


Questo \`e il primo capitolo.
\section{Funzionamento}                 %crea la sezione
Questa \`e la prima sezione.

Ora vediamo un elenco numerato:         %crea un elenco numerato
\begin{enumerate}
\item primo oggetto
\item secondo oggetto
\item terzo oggetto
\item quarto oggetto
\end{enumerate}

\begin{figure}[h]                       %crea l'ambiente figura; [h] sta
                                        %   per here, cio� la figura va qui
\begin{center}                          %centra nel mezzo della pagina
                                        %   la figura
%\includegraphics[width=5cm]{figura.eps}%inserisce una figura larga 5cm
                                        %se si vuole usare va scommentata
%
%%%%%%%%%%%%%%%%%%%%%%%%%%%%%%%%%%%%%%%%%inserisce la legenda ed etichetta
                                        %   la figura con \label{fig:prima}
\caption[legenda elenco figure]{legenda sotto la figura}\label{fig:prima}
\end{center}
\end{figure}

\section{Seconda Sezione}
Ora vediamo un elenco puntato:
\begin{itemize}                         %crea un elenco puntato
\item primo oggetto
\item secondo oggetto
\end{itemize}

\section{Altra Sezione}
Vediamo un elenco descrittivo:
\begin{description}                     %crea un elenco descrittivo
  \item[OGGETTO1] prima descrizione;
  \item[OGGETTO2] seconda descrizione;
  \item[OGGETTO3] terza descrizione.
\end{description}
%%%%%%%%%%%%%%%%%%%%%%%%%%%%%%%%%%%%%%%%%crea una sottosezione
\subsection{Altra SottoSezione}
%%%%%%%%%%%%%%%%%%%%%%%%%%%%%%%%%%%%%%%%%crea una sottosottosezione
\subsubsection{SottoSottoSezione}Questa sottosottosezione non viene
numerata, ma \`e solo scritta in grassetto.
\section{Altra Sezione}                 %crea una sottosezione
Vediamo la creaz-italian non precedentemente selezionato.
(Lettura del database... 233368 file ione di una tabella; la tabella \ref{tab:uno}
(richiamo il nome della tabella utilizzando la label che ho messo sotto):
la facciamo di tre righe e tre colonne, la prima colonna
``incolonnata'' a destra (r) e le altre centrate (c):\\
\begin{table}[h]                        %ambiente tabella
                                        %(serve per avere la legenda)
\begin{center}                          %centra nella pagina la tabella
\begin{tabular}{r|c|c}                  %tre colonne con righe verticali
                                        %   prodotte con |
\hline \hline                           %inserisce due righe orizzontali
$(1,1)$ & $(1,2)$ & $(1,3)$\\           %& separa le colonne e con
\hline                                  %inserisce una riga orizzontale
$(2,1)$ & $(2,2)$ & $(2,3)$\\           %  \\ va a capo
\hline                                  %inserisce una riga orizzontale
$(3,1)$ & $(3,2)$ & $(3,3)$\\
\hline \hline                           %inserisce due righe orizzontali
\end{tabular}
\caption[legenda elenco tabelle]{legenda tabella}\label{tab:uno}
\end{center}
\end{table}
\section{Altra Sezione}\label{sec:prova}%posso mettere le label anche
                                        %   alle section
\subsection{Listati dei programmi}
\subsubsection{Primo Listato}
\begin{verbatim}
        In questo ambiente     posso scrivere      come voglio,
lasciare gli spazi che voglio e non % commentare quando voglio
e ci sar� scritto tutto.
Quando lo uso � meglio che disattivi il Wrap del WinEdt
\end{verbatim}
%%%%%%%%%%%%%%%%%%%%%%%%%%%%%%%%%%%%%%%%%non numera l'ultima pagina sinistra
\clearpage{\pagestyle{empty}\cleardoublepage}

\chapter{Casi d'uso}                %crea il capitolo
%%%%%%%%%%%%%%%%%%%%%%%%%%%%%%%%%%%%%%%%%imposta l'intestazione di pagina
\lhead[\fancyplain{}{\bfseries\thepage}]{\fancyplain{}{\bfseries\rightmark}}


Questo \`e il primo capitolo.
\section{esempi}                 %crea la sezione
Questa \`e la prima sezione.

Ora vediamo un elenco numerato:         %crea un elenco numerato
\begin{enumerate}
\item primo oggetto
\item secondo oggetto
\item terzo oggetto
\item quarto oggetto
\end{enumerate}

\begin{figure}[h]                       %crea l'ambiente figura; [h] sta
                                        %   per here, cio� la figura va qui
\begin{center}                          %centra nel mezzo della pagina
                                        %   la figura
%\includegraphics[width=5cm]{figura.eps}%inserisce una figura larga 5cm
                                        %se si vuole usare va scommentata
%
%%%%%%%%%%%%%%%%%%%%%%%%%%%%%%%%%%%%%%%%%inserisce la legenda ed etichetta
                                        %   la figura con \label{fig:prima}
\caption[legenda elenco figure]{legenda sotto la figura}\label{fig:prima}
\end{center}
\end{figure}

\section{Seconda Sezione}
Ora vediamo un elenco puntato:
\begin{itemize}                         %crea un elenco puntato
\item primo oggetto
\item secondo oggetto
\end{itemize}

\section{Altra Sezione}
Vediamo un elenco descrittivo:
\begin{description}                     %crea un elenco descrittivo
  \item[OGGETTO1] prima descrizione;
  \item[OGGETTO2] seconda descrizione;
  \item[OGGETTO3] terza descrizione.
\end{description}
%%%%%%%%%%%%%%%%%%%%%%%%%%%%%%%%%%%%%%%%%crea una sottosezione
\subsection{Altra SottoSezione}
%%%%%%%%%%%%%%%%%%%%%%%%%%%%%%%%%%%%%%%%%crea una sottosottosezione
\subsubsection{SottoSottoSezione}Questa sottosottosezione non viene
numerata, ma \`e solo scritta in grassetto.
\section{Altra Sezione}                 %crea una sottosezione
Vediamo la creaz-italian non precedentemente selezionato.
(Lettura del database... 233368 file ione di una tabella; la tabella \ref{tab:uno}
(richiamo il nome della tabella utilizzando la label che ho messo sotto):
la facciamo di tre righe e tre colonne, la prima colonna
``incolonnata'' a destra (r) e le altre centrate (c):\\
\begin{table}[h]                        %ambiente tabella
                                        %(serve per avere la legenda)
\begin{center}                          %centra nella pagina la tabella
\begin{tabular}{r|c|c}                  %tre colonne con righe verticali
                                        %   prodotte con |
\hline \hline                           %inserisce due righe orizzontali
$(1,1)$ & $(1,2)$ & $(1,3)$\\           %& separa le colonne e con
\hline                                  %inserisce una riga orizzontale
$(2,1)$ & $(2,2)$ & $(2,3)$\\           %  \\ va a capo
\hline                                  %inserisce una riga orizzontale
$(3,1)$ & $(3,2)$ & $(3,3)$\\
\hline \hline                           %inserisce due righe orizzontali
\end{tabular}
\caption[legenda elenco tabelle]{legenda tabella}\label{tab:uno}
\end{center}
\end{table}
\section{Altra Sezione}\label{sec:prova}%posso mettere le label anche
                                        %   alle section
\subsection{Listati dei programmi}
\subsubsection{Primo Listato}
\begin{verbatim}
        In questo ambiente     posso scrivere      come voglio,
lasciare gli spazi che voglio e non % commentare quando voglio
e ci sar� scritto tutto.
Quando lo uso � meglio che disattivi il Wrap del WinEdt
\end{verbatim}
%%%%%%%%%%%%%%%%%%%%%%%%%%%%%%%%%%%%%%%%%non numera l'ultima pagina sinistra
\clearpage{\pagestyle{empty}\cleardoublepage}
%%%%%%%%%%%%%%%%%%%%%%%%%%%%%%%%%%%%%%%%%per fare le conclusioni


\chapter*{Conclusioni}
%%%%%%%%%%%%%%%%%%%%%%%%%%%%%%%%%%%%%%%%%imposta l'intestazione di pagina
\rhead[\fancyplain{}{\bfseries
CONCLUSIONI}]{\fancyplain{}{\bfseries\thepage}}
\lhead[\fancyplain{}{\bfseries\thepage}]{\fancyplain{}{\bfseries
CONCLUSIONI}}
%%%%%%%%%%%%%%%%%%%%%%%%%%%%%%%%%%%%%%%%%aggiunge la voce Conclusioni
                                        %   nell'indice
\addcontentsline{toc}{chapter}{Conclusioni} Queste sono le
conclusioni.\\
In queste conclusioni voglio fare un riferimento alla
bibliografia: questo \`e il mio riferimento \cite{K3,K4}.
%%%%%%%%%%%%%%%%%%%%%%%%%%%%%%%%%%%%%%%%%imposta l'intestazione di pagina
\renewcommand{\chaptermark}[1]{\markright{\thechapter \ #1}{}}
\lhead[\fancyplain{}{\bfseries\thepage}]{\fancyplain{}{\bfseries\rightmark}}
\appendix                               %imposta le appendici
\chapter{Prima Appendice}               %crea l'appendice
In questa Appendice non si \`e utilizzato il comando:\\
%%%%%%%%%%%%%%%%%%%%%%%%%%%%%%%%%%%%%%%%%\verb"" � equivalente all'
                                        %   ambiente verbatim,
                                        %   ma si utilizza all'interno
                                        %   di un discorso.
\verb"\clearpage{\pagestyle{empty}\cleardoublepage}", ed infatti
l'ultima pagina 8 ha l'intestazione con il numero di pagina in
alto.
%%%%%%%%%%%%%%%%%%%%%%%%%%%%%%%%%%%%%%%%%imposta l'intestazione di pagina
\rhead[\fancyplain{}{\bfseries \thechapter \:Prima Appendice}]
{\fancyplain{}{\bfseries\thepage}}
\chapter{Seconda Appendice}             %crea l'appendice
%%%%%%%%%%%%%%%%%%%%%%%%%%%%%%%%%%%%%%%%%imposta l'intestazione di pagina
\rhead[\fancyplain{}{\bfseries \thechapter \:Seconda Appendice}]
{\fancyplain{}{\bfseries\thepage}}
\begin{thebibliography}{90}             %crea l'ambiente bibliografia
\rhead[\fancyplain{}{\bfseries \leftmark}]{\fancyplain{}{\bfseries
\thepage}}
%%%%%%%%%%%%%%%%%%%%%%%%%%%%%%%%%%%%%%%%%aggiunge la voce Bibliografia
                                        %   nell'indice
\addcontentsline{toc}{chapter}{Bibliografia}
%%%%%%%%%%%%%%%%%%%%%%%%%%%%%%%%%%%%%%%%%provare anche questo comando:
%%%%%%%%%%%\addcontentsline{toc}{chapter}{\numberline{}{Bibliografia}}
\bibitem{K1} Renzo Davoli. Internet of Threads. http://www.cs.unibo.it/~renzo/papers/2013.iciw.pdf, 2013
\bibitem{K2} Renzo Davoli. Internet of Threads: Processes as Internet Nodes. http://www.cs.unibo.it/~renzo/papers/2014.IntTechIoTh.pdf, 2014
\bibitem{K3} Renzo Davoli. Internet of Threads. http://www.cs.unibo.it/~renzo/papers/ConfGARR11\_SelectedPapers\_Davoli.pdf
\bibitem{K4} Altran. picoTCP. https://github.com/tass-belgium/picotcp
\bibitem{K5} Virtual Square Team. LWIPv6. http://wiki.v2.cs.unibo.it/wiki/index.php/LWIPV6
\bibitem{K6} Virtual Square Team. UMview. http://wiki.v2.cs.unibo.it/wiki/index.php/UMview
\bibitem{K7} Virtual Square Team. purelibc. http://wiki.virtualsquare.org/wiki/index.php/PureLibc
\end{thebibliography}
%%%%%%%%%%%%%%%%%%%%%%%%%%%%%%%%%%%%%%%%%non numera l'ultima pagina sinistra
\clearpage{\pagestyle{empty}\cleardoublepage}
\chapter*{Ringraziamenti}
\thispagestyle{empty}
Qui possiamo ringraziare il mondo intero!!!!!!!!!!\\
Ovviamente solo se uno vuole, non \`e obbligatorio.
\end{document}
