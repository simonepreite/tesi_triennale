

%%%%%%%%%%%%%%%%%%%%%%%%%%%%%%%%%%%%%%%%%12pt: grandezza carattere
                                        %a4paper: formato a4
                                        %openright: apre i capitoli a destra
                                        %twoside: serve per fare un
                                        %   documento fronteretro
                                        %report: stile tesi (oppure book)
\documentclass[12pt,a4paper,openright,twoside]{report}
%
%%%%%%%%%%%%%%%%%%%%%%%%%%%%%%%%%%%%%%%%%libreria per scrivere in italiano
\usepackage[italian]{babel}
%
%%%%%%%%%%%%%%%%%%%%%%%%%%%%%%%%%%%%%%%%%libreria per accettare i caratteri
                                        %   digitati da tastiera come � �
                                        %   si pu� usare anche
                                        %   \usepackage[T1]{fontenc}
                                        %   per� con questa libreria
                                        %   il tempo di compilazione
                                        %   aumenta
\usepackage{newlfont}

\usepackage[latin1]{inputenc}
%%\usepackage[utf8]{inputenc}

%
%%%%%%%%%%%%%%%%%%%%%%%%%%%%%%%%%%%%%%%%%libreria per impostare il documento
\usepackage{fancyhdr}
%
%%%%%%%%%%%%%%%%%%%%%%%%%%%%%%%%%%%%%%%%%libreria per avere l'indentazione
%%%%%%%%%%%%%%%%%%%%%%%%%%%%%%%%%%%%%%%%%   all'inizio dei capitoli, ...
\usepackage{indentfirst}
%%%%%%%%%%%%%%%%%%%%%%%%%%%%%%%%%%%%%%%%%libreria per inserire grafici
\usepackage{graphicx}
%
%%%%%%%%%%%%%%%%%%%%%%%%%%%%%%%%%%%%%%%%%libreria per utilizzare font
                                        %   particolari ad esempio
                                        %   \textsc{}
\graphicspath{{image/}}
\usepackage{newlfont}
%
%%%%%%%%%%%%%%%%%%%%%%%%%%%%%%%%%%%%%%%%%librerie matematiche
\usepackage{amssymb}
\usepackage{amsmath}
\usepackage{latexsym}
\usepackage{amsthm}
%\usepackage[breaklinks=true]{hyperref}
\usepackage{hyperref}
\usepackage{subfiles}
\usepackage{xcolor}
\usepackage{subfigure}
\usepackage{listings}

\definecolor{mGreen}{rgb}{0,0.6,0}
\definecolor{mGray}{rgb}{0.5,0.5,0.5}
\definecolor{mPurple}{rgb}{0.58,0,0.82}
\definecolor{backgroundColour}{rgb}{0.95,0.95,0.92}

\lstdefinestyle{CStyle}{
belowcaptionskip=1\baselineskip,
backgroundcolor=\color{backgroundColour},
  breaklines=true,
  breakatwhitespace=false,
  language=C,
  showstringspaces=false,
  basicstyle=\ttfamily\small,
  keywordstyle=\bfseries\color{green!40!black},
  commentstyle=\itshape\color{purple!40!black},
  morekeywords=[2]{fill_buf_route,fill_buf_addr, handle_vsnlib, fill_buf_link, init_vsnlib, adddellink, getsetlink, adddeladdr, getaddr, adddelroute, getroute, rtm_newroute_c, rtm_delroute_c, rtm_getroute_c, rtm_newaddr_c, rtm_deladdr_c, rtm_getaddr_c, rtm_newlink_c, rtm_dellink_c, rtm_getlink_c, rtm_setlink_c},  keywordstyle=[2]\color{blue},
  identifierstyle=\color{black},
  stringstyle=\color{orange},
  tabsize=2
}

\lstdefinestyle{CscriptStyle}{
belowcaptionskip=1\baselineskip,
backgroundcolor=\color{backgroundColour},
  breaklines=true,
  breakatwhitespace=false,
  language=C,
  showstringspaces=false,
  basicstyle=\ttfamily\scriptsize,
  keywordstyle=\bfseries\color{green!40!black},
  commentstyle=\itshape\color{purple!40!black},
  keywordstyle=[2]\color{blue},
  morekeywords=[2]{fill_buf_route,fill_buf_addr, handle_vsnlib, fill_buf_link, init_vsnlib, adddellink, getsetlink, adddeladdr, getaddr, adddelroute, getroute, rtm_newroute_c, rtm_delroute_c, rtm_getroute_c, rtm_newaddr_c, rtm_deladdr_c, rtm_getaddr_c, rtm_newlink_c, rtm_dellink_c, rtm_getlink_c, rtm_setlink_c},
  identifierstyle=\color{black},
  stringstyle=\color{orange},
  keywordstyle=[4]\color{brown},
  morekeywords=[4]{IP64_ADDR, IP64_MASKADDR, NLMSG_NEXT, NLMSG_DATA, NLMSG_DONE, NLMSG_OK},
  keywordstyle=[5]\color{red},
  morekeywords=[5]{API_TABLE_SIZE, RTDL_LAZY, RTM_BASE, NULL, INET6_ADDRSTRLEN, AF_INET, AF_INET6, RTM_NEWADDR, RTM_DELADDR, IP_ADD_ANY, RTM_NEWROUTE, RTM_DELROUTE, PF_INET, PF_INET6},
  tabsize=2
}


\lstdefinelanguage{Bashn}[]{bash}{
backgroundcolor=\color{backgroundColour},
commentstyle=\color{mGreen},
keywordstyle=\color{magenta},
keywordstyle=[3]\color{blue},
keywordstyle=[2]\color{red},
keywords={mkdir, git, vde_switch, vdens, rm, cd, cmake, make, sudo, echo, ip, nc, ./testX.exe},
keywords=[3]{addr, link, route},
keywords=[2]{CC, FLAG_PREP, FLAG_CC, BUILDDIR, OBJ, STACK, SWITCH},
numberstyle=\tiny\color{mGray},
stringstyle=\color{mPurple},
basicstyle=\ttfamily\small,
breakatwhitespace=false,
breaklines=true,
captionpos=b,
keepspaces=true,
numbers=left,
numbersep=5pt,
showspaces=false,
showstringspaces=false,
showtabs=false,
tabsize=2,
moredelim=[is][\color{blue}]{[}{]}
}

%
\oddsidemargin=30pt \evensidemargin=20pt%impostano i margini
\hyphenation{sil-la-ba-zio-ne pa-ren-te-si}%serve per la sillabazione: tra parentesi
					   %vanno inserite come nell'esempio le parole
%					   %che latex non riesce a tagliare nel modo giusto andando a capo.

%
%%%%%%%%%%%%%%%%%%%%%%%%%%%%%%%%%%%%%%%%%comandi per l'impostazione
                                        %   della pagina, vedi il manuale
                                        %   della libreria fancyhdr
                                        %   per ulteriori delucidazioni
\pagestyle{fancy}\addtolength{\headwidth}{20pt}
\renewcommand{\chaptermark}[1]{\markboth{\thechapter.\ #1}{}}
\renewcommand{\sectionmark}[1]{\markright{\thesection \ #1}{}}
\rhead[\fancyplain{}{\bfseries\leftmark}]{\fancyplain{}{\bfseries\thepage}}
\cfoot{}
%%%%%%%%%%%%%%%%%%%%%%%%%%%%%%%%%%%%%%%%%
\linespread{1.3}                        %comando per impostare l'interlinea
%%%%%%%%%%%%%%%%%%%%%%%%%%%%%%%%%%%%%%%%%definisce nuovi comandi
%
\begin{document}
\begin{titlepage}
\begin{center}
{{\Large{\textsc{Alma Mater Studiorum $\cdot$ Universit\`a di
Bologna}}}} \rule[0.1cm]{13.8cm}{0.1mm}
\rule[0.5cm]{13.8cm}{0.6mm}
{\bf SCUOLA DI SCIENZE\\
Corso di Laurea in Informatica }
\end{center}
\vspace{25mm}
\begin{center}
  {\large{\bf VSNLib: UN' INTERFACCIA UNICA}}\\
  \vspace{2mm}
  {\large{\bf DI CONFIGURAZIONE PER STACK VIRTUALI}}\\
  \vspace{2mm}
  {\large{\bf TRAMITE PACCHETTI NETLINK}}\\
  \end{center}
  \vspace{30mm}
  \par
  \noindent
  \begin{minipage}[t]{0.5\textwidth}
  {\large{\bf Relatore:\\
  Chiar.mo Prof.\\
  RENZO DAVOLI}}\\
  \end{minipage}
  \hfill
  \begin{minipage}[t]{0.5\textwidth}\raggedleft
  {\large{\bf Presentata da:\\
  SIMONE PREITE}}
\end{minipage}
\vspace{30mm}
\begin{center}
{\large{\bf Sessione II\\%inserire il numero della sessione in cui ci si laurea
Anno Accademico 2016/17}}%inserire l'anno accademico a cui si è iscritti
\end{center}
\end{titlepage}

%%\begin{titlepage}
\begin{center}
{{\Large{\textsc{Alma Mater Studiorum $\cdot$ Universit\`a di
Bologna}}}} \rule[0.1cm]{13.8cm}{0.1mm}
\rule[0.5cm]{13.8cm}{0.6mm}
{\bf SCUOLA DI SCIENZE\\
Corso di Laurea in Informatica }
\end{center}
\vspace{25mm}
\begin{center}
  {\large{\bf VSNLib: UN' INTERFACCIA UNICA}}\\
  \vspace{2mm}
  {\large{\bf DI CONFIGURAZIONE PER STACK VIRTUALI}}\\
  \vspace{2mm}
  {\large{\bf TRAMITE PACCHETTI NETLINK}}\\
  \end{center}
  \vspace{30mm}
  \par
  \noindent
  \begin{minipage}[t]{0.5\textwidth}
  {\large{\bf Relatore:\\
  Chiar.mo Prof.\\
  RENZO DAVOLI}}\\
  \end{minipage}
  \hfill
  \begin{minipage}[t]{0.5\textwidth}\raggedleft
  {\large{\bf Presentata da:\\
  SIMONE PREITE}}
\end{minipage}
\vspace{30mm}
\begin{center}
{\large{\bf Sessione II\\%inserire il numero della sessione in cui ci si laurea
Anno Accademico 2016/17}}%inserire l'anno accademico a cui si è iscritti
\end{center}
\end{titlepage}

\newpage
\null
\thispagestyle{empty}
\newpage

\begin{titlepage}                       %crea un ambiente libero da vincoli
                                        %   di margini e grandezza caratteri:
                                        %   si pu\`o modificare quello che si
                                        %   vuole, tanto fuori da questo
                                        %   ambiente tutto viene ristabilito
%

\thispagestyle{empty}                   %elimina il numero della pagina
\topmargin=6.5cm                        %imposta il margina superiore a 6.5cm
\raggedleft                             %incolonna la scrittura a destra
\large                                  %aumenta la grandezza del carattere
                                        %   a 14pt
\em                                     %emfatizza (corsivo) il carattere
Questa \`e la \textsc{Dedica}:\\
ognuno pu\`o scrivere quello che vuole, \\
anche nulla \ldots                      %\ldots lascia tre puntini
\end{titlepage}

\newpage
\newpage

%
%%%%%%%%%%%%%%%%%%%%%%%%%%%%%%%%%%%%%%%%
\clearpage{\pagestyle{empty}\cleardoublepage}%non numera l'ultima pagina sinistra
\pagenumbering{roman}                   %serve per mettere i numeri romani



\chapter*{Introduzione}                 %crea l'introduzione (un capitolo
                                        %   non numerato)

%%%%%%%%%%%%%%%%%%%%%%%%%%%%%%%%%%%%%%%%%imposta l'intestazione di pagina
\rhead[\fancyplain{}{\bfseries
INTRODUZIONE}]{\fancyplain{}{\bfseries\thepage}}
\lhead[\fancyplain{}{\bfseries\thepage}]{\fancyplain{}{\bfseries
INTRODUZIONE}}
%%%%%%%%%%%%%%%%%%%%%%%%%%%%%%%%%%%%%%%%%aggiunge la voce Introduzione
                                        %   nell'indice
\addcontentsline{toc}{chapter}{Introduzione}

Nell'ambito delle reti negli ultimi anni con l'avvento di internet si \`e assistito ad una crescita esponenziale delle apparecchiature connesse a questa rete.
Fino ad oggi questa immensa rete \`e stata gestita per lo pi\`u attraverso indirizzamento delle interfacce di rete di cui i computer sono dotati.\\
Al momento per\`o ci si trova a fare i conti con la carenza degli indirizzi e pian piano si procede verso una migrazione basata sul nuovo (ma non troppo) protocollo IPv6 che prevede uno spazio di indirizzamento molto pi\`u ampio che lascia spazio a nuove idee impraticabili con protocollo IPv4.\\
In seguito sentiremo parlare di IoT (Internet of Things) ed IoTh (Internet of Threads), due paradigmi che possiamo affermare essere il risultato di un'evoluzione sempre crescente dei sistemi informativi.\\
L'elaborato sarà strutturato in capitoli.\\
Il primo di questi si premura di introdurre il lettore allo stato attuale dei protolli di rete e del paradigma IoTh; nel secondo capitolo verr\`a trattato il progetto VsnLib, ovvero la libreria oggetto della tesi; in fine il terzo capito si occuper\`a di descrivere alcuni esempi di utilizzo della stessa con immagini dimostrative ed istruzioni.
%%%%%%%%%%%%%%%%%%%%%%%%%%%%%%%%%%%%%%%%%non numera l'ultima pagina sinistra
\clearpage{\pagestyle{empty}\cleardoublepage}
\tableofcontents                        %crea l'indice
%%%%%%%%%%%%%%%%%%%%%%%%%%%%%%%%%%%%%%%%%imposta l'intestazione di pagina
\rhead[\fancyplain{}{\bfseries\leftmark}]{\fancyplain{}{\bfseries\thepage}}
\lhead[\fancyplain{}{\bfseries\thepage}]{\fancyplain{}{\bfseries
INDICE}}
%%%%%%%%%%%%%%%%%%%%%%%%%%%%%%%%%%%%%%%%%non numera l'ultima pagina sinistra
\clearpage{\pagestyle{empty}\cleardoublepage}
\listoffigures                          %crea l'elenco delle figure
%%%%%%%%%%%%%%%%%%%%%%%%%%%%%%%%%%%%%%%%%non numera l'ultima pagina sinistra
\clearpage{\pagestyle{empty}\cleardoublepage}
\listoftables                           %crea l'elenco delle tabelle
%%%%%%%%%%%%%%%%%%%%%%%%%%%%%%%%%%%%%%%%%non numera l'ultima pagina sinistra
\clearpage{\pagestyle{empty}\cleardoublepage} 



\chapter{Internet of Threads}                %crea il capitolo
%%%%%%%%%%%%%%%%%%%%%%%%%%%%%%%%%%%%%%%%%imposta l'intestazione di pagina
\lhead[\fancyplain{}{\bfseries\thepage}]{\fancyplain{}{\bfseries\rightmark}}
\pagenumbering{arabic}                  %mette i numeri arabi
\section{Definizione}                 %crea la sezione
L'Internet of Things, ovvero l'internet delle cose, vuole rappresentare la diffusione dei sistemi embedded come veri e propri nodi di reti. Nell'ottica dell'IoT, oggi, apparecchi comununi in un'abitazione, quali condizionatore, la caldaia o addirittura il tostapane, possono essere nodi di rete ed essere quindi configurati e controllati attraverso di essa.\\
Il concetto di Internet of Threads \`e una evoluzione di questa astrazione. L'idea alla base di IoTh \`e quella di avere non solo oggetti ma anche processi come nodi di rete.\\
Per poter capire pienamente i vantaggi dell'IoTh si pu\`o pensare ad un'analogia con quanto \`e successo nel mondo della telefonia. In passato la telefonia collegava apparecchi fissi connessi via cavo. In questo modello il numero di telefono individuava un luogo e non una persona e quindi era comune, per trovare una persona, doversi interrogare sul dove questa potesse essere; al contrario l'avvento dei telefoni cellulari ha cambiato l'idea di significato di numero di telefono, non pi\`u associato a luoghi fisici ma ad una persona infatti i telefoni cellulari sono normalmente telefoni personali\cite{K1,K2}.\\
Come possiamo notare infatti, con questa tecnologia lo stack \`e direttamente integrato nel programma.
\begin{figure}[h]
     \begin{center}%
        \subfigure[stack associato all'interfaccia fisica]{%
            \label{fig:physical}
            \includegraphics[width=7cm]{old_stack}
        }%
        \subfigure[stack associato ai processi]{%
           \label{fig:virtual}
           \includegraphics[width=7cm]{new_stack}
        }
%
    \end{center}
    \caption{%
        Differenze tra stack reali e virtuali
     }%
\end{figure}%\
Nello stesso modo IoTh indirizza i processi che forniscono un servizio, significa che questi processi non sono pi\`u vincolati a risiedere in un luogo specifico, o in un elaboratore specifico nell'ambito delle reti, ma \`e possibile trasparentemente farli migrare da un elaboratore all'altro senza che questo crei difficolt\`a di indirizzamento.\\
Una nota va anche fatta in termini di sicurezza. La possibilit\`a che ogni servizio utilizzi propri indirizzi IP e uno stack di rete privato rende molto pi\`u difficile se non praticamente impossibile sfruttare le fragilit\`a di un servizio per poter scalare l'intrusione ad altri servizi presenti su di esso.

\section{Stack Implementati}
Diversi sono i progetti che si occupano di offrire ai processi il proprio stack di rete; tra questi ne verranno presi in esame tre, considerati pi\`u indicati per uno studio in quanto open source. Ognuno di essi offre la propria implementazione di stack di rete.\\
Uno stack di rete legato al processo permette di connettere lo stesso ad una rete reale, tramite le interfacce fornite dal sistema, o virtuale (ad esempio VDE) come se fosse una macchina fisica a se stante, a questo punto ogni processo pu\`o staccarsi dal modo in cui la macchina che lo ospita gestisce la rete ed avere le proprie regole.\\

\subsection{PicoTCP}
PicoTCP\cite{K4} \`e supportato da Tass Belgium (Altran); \`e lo stack pi\`u conosciuto e diffuso per sistemi embedded ed esistono anche dei progetti basati su IP di reti mesh realizzati con picoTCP\cite{K14}.\\
PicoTCP presenta una struttura modulare (schema molto utilizzato per questo tipo di progetto come vedremo in seguito), che permette di avere il minimo indispensabile all'interno dell'applicazione.\\
Vanta la compatibilit\`a con un gran numero di dispositivi costruiti con diversi processori e interfacce di rete, pu\`o essere compilato con diversi compilatori ed \`e gi\`a integrato in alcuni sistemi nati per ridurre al minimo il consumo di risorse.\\
Purtroppo a causa di incompatibilit\`a con la licenza GPL alcune parti del codice non sono state rese pubbliche anche se rappresentano una piccola percentuale dell'intero progetto.
\subsection{LWIP}
Nasce come progetto di Adam Dunkels pensato per sistemi embedded ma che inizialmente non forniva supporto ad IPv6.\\
Light-weight IP (LWIP\cite{K13}) vanta un core molto piccolo e la possibilit\`a di eseguire anche senza sistema oprativo (single-threaded) e con il minor consumo di RAM possibile, inoltre \`e modulare ed offre la possibilit\`a di aggiungere protocolli come DHCP e DNS solo in caso di necessit\`a.\\
Supporta sia little che big endian e gira su processori a 8 e 32 bit, quindi funziona anche su un semplice ATMEGA.\\
Unica pecca \`e il protoccolo IPv6 che \`e ancora in fase sperimentale, infatti il supporto IPv6 pu\`o essere aggiunto ed \`e scaricabile da git.
\subsection{LWIPv6}
LWIPv6 nasce per colmare la mancanza del supporto IPv6 in LWIP, ed \`e questo uno dei motivi della nascita di LWIPv6\cite{K6}.\\
\begin{description}                     %crea un elenco descrittivo
  \item[virtualsquare\cite{K15}: ] un progetto ideato da Renzo Davoli e Michael Goldweber. Comprende una gamma di tool pensati per sperimentare e creare strumenti di uso comune in tema di virtualizzazione.
\end{description}
Quando al team di virtualsquare \`e sopravvenuta la necessit\`a di avere uno stack per la rete vde non esisteva ancora nulla di maturo o comunque plug and play, da questa esigenza il team ha pensato di realizzare uno stack versatile in versione libreria tale per cui ogni processo potesse avere il proprio stack di rete.\\
Caratteristica principale di LWIPv6 \`e il suo motore ibrido che, di fatto funziona solo con IPv6 e mappa in questo protocollo le comunicazioni IPv4.\\
La conversione degli indirizzi ip da IPv4 ad IPv6 \`e particolare ma intuitiva. I 32 bit finali dell'indirizzo IPv6 rappresentano quello in versione 4, i 16 bit antecedenti sono settati a 1 e la parte restante, i primi 80 bit, a 0. Di seguito un esmpio sull'indirizzo 192.168.1.2:
\begin{table}[h]                        %ambiente tabella
                                        %(serve per avere la legenda)
\begin{center}                          %centra nella pagina la tabella
\begin{tabular}{r|c|c|c}                  %tre colonne con righe verticali
                                        %   prodotte con |
\hline
$IP version$ & $80 bit$ & $16 bit$ & $32 bit$\\
\hline \hline                         %inserisce due righe orizzontali
$IPv4$ & $unused$  & $unused$  & $192.168.1.2$\\           %& separa le colonne e con
\hline                                  %inserisce una riga orizzontale
$IPv6$ & $00000000.00000000.0000$ & $FFFF$ & $192.168.1.2$\\           %  \\ va a capo
\hline                                  %inserisce una riga orizzontale
$IPv6$ & $00000000.00000000.0000$ & $FFFF$ & $C0A80102$\\
\hline                           %inserisce due righe orizzontali
\end{tabular}
\caption[IPv4 to IPv6 conversion]{rappresentazione di un indirizzo IPv4 in IPv6}\label{tab:IPv4toIPv6}
\end{center}
\end{table}

Per quanto riguarda le maschere per\`o questa configurazione non pu\`o essere adottata. Se i primi 80 bit venissero settati a 0 non sarebbero considerati e nel caso in cui due reti differissero proprio nei primi 80 bit non potremmo pi\`u smistare i pacchetti verso la corretta rete di destinazione.\\
LWIPv6 si caratterizza perch\`e tramite una modifica della conversione non ha bisogno di usare due motori diversi per IPv4 ed IPv6 ma uno che \`e ibrido. La modifica \`e semplice ed \`e stato sufficiente ribaltare i primi 80 bit. Come prima suggeriamo un esempio di conversione di una maschera all'interno di LWIPv6:

\begin{table}[h]                        %ambiente tabella
                                        %(serve per avere la legenda)
\begin{center}                          %centra nella pagina la tabella
\begin{tabular}{r|c|c|c}                  %tre colonne con righe verticali
                                        %   prodotte con |
\hline
$IP version$ & $80 bit$ & $16 bit$ & $32 bit$\\
\hline  \hline        %inserisce due righe orizzontali
$IPv4$ & $unused$  &  $unused$  & $255.255.255.0$\\           %& separa le colonne e con
\hline                                  %inserisce una riga orizzontale
$IPv6$ & $FFFFFFFF.FFFFFFFF.FFFF$ & $FFFF$ & $255.255.255.0$\\           %  \\ va a capo
\hline                                  %inserisce una riga orizzontale
$IPv6$ & $FFFFFFFF.FFFFFFFF.FFFF$ & $FFFF$ & $FFFFFF00$\\
\hline                           %inserisce due righe orizzontali
\end{tabular}
\caption[IPv4 to IPv6 mask conversion]{rappresentazione delle maschere in LWIPv6}\label{tab:IPv4toIPv6 mask}
\end{center}
\end{table}
\section{Netlink}
\subsection{Inter Process Communication(IPC)}
Molti processi necessitano di scambiare informazioni per i motivi pi\`u disparati, dalla comunicazione di rete a quella interna ad una sola macchina.\\
\`E banalmente limitativo pensare di costruire solo programmi fini a se stessi e che quindi non comprendano nessuna interazione con altri programmi.\\
Con IPC intendiamo quindi l'insieme delle tecnologie adottate per permettere ai processi di comunicare tra essi, che siano ospitati sulla stessa macchina o distribuiti sulla rete, tra queste tecnologie esiste appunto quella utilizzata all'interno della libreria VsnLib di cui parleremo nello specifico del prossimo capitolo..
\subsection{Il Sistema IPC Netlink}
Netlink \`e un sistema IPC adottato dal kernel linux perch\`e in grado di mettere in comunicazione diversi task, solitamente serve per far comunicare task in user-space con task in kernel-space ma pu\`o far interagire anche processi entrambi in user-space.\\
Studiato per essere il successore di ioctl per le configurazioni ed il monitoraggio, si propone di essere pi\`u flessibile.
Ma come avviene esattamente la comunicazione attraverso netlink?\\
Netlink utilizza un sistema di socket indicizzato in base ai processi ogni processo pu\`o definire socket diversi di tipo netlink (AF\_NETLINK, NETLINK\_GENERIC, NETLINK\_XFRM) per inviare e ricevere messaggi con e da altri processi, implementa un sistema di porte basato sul process ID dei processi; ad esempio per la comunicazione con il kernel viene usata la porta 0.\\
L'immagine successiva rende perfettamente il concetto.
\begin{figure}[h]                       %crea l'ambiente figura; [h] sta
                                        %   per here, cio� la figura va qui
\begin{center}                          %centra nel mezzo della pagina
                                        %   la figura
\includegraphics[width=10cm]{netlink_comunication}%inserisce una figura larga 5cm
                                        %se si vuole usare va scommentata
%
%%%%%%%%%%%%%%%%%%%%%%%%%%%%%%%%%%%%%%%%%inserisce la legenda ed etichetta
                                        %   la figura con \label{fig:prima}
\caption[comunicazione netlink]{scambio di messaggi attraverso socket netlink}
\end{center}
\end{figure}\\
L'interfaccia di comunicazione \`e abbastanza standardizzata ma il payload \`e personalizzabile ed \`e possibile definire la propria struttura da utilizzare per inviare dati (informazioni), tra processi diversi, attraverso i socket netlink.
\begin{figure}[h]                       %crea l'ambiente figura; [h] sta
                                        %   per here, cio� la figura va qui
\begin{center}                          %centra nel mezzo della pagina
                                        %   la figura
\includegraphics[width=8cm]{nlmsghdr}%inserisce una figura larga 5cm
                                        %se si vuole usare va scommentata
%
%%%%%%%%%%%%%%%%%%%%%%%%%%%%%%%%%%%%%%%%%inserisce la legenda ed etichetta
                                        %   la figura con \label{fig:prima}
\caption[struct nlmsghdr]{schema della struttura nlmsghdr}
\end{center}
\end{figure}\\
\subsection{Libnl}
Netlink Protocol Library Suite (libnl) \cite{K10}, \`e una raccolta di librerie ed utility che forniscono API di comunicazione netlink basate su quelle del kernel linux e comprende:
\begin{description}                     %crea un elenco descrittivo
  \item[libnl] Core della libreria, offre uno strato di unificazione delle interfacce sulle quali poi si basano le altre librerie che pertanto di pendo da questa;
  \item[libnl-route] Questa libreria si occupa di fornire API per la configurazione degli elemnti della famiglia NETLINK\_ROUTE;
  \item[libnl-genl] genl significa generic netlink e questa libreria offre una versione estesa del protocollo netlink;
  \item[libnl-nf] API per configurazioni netlink basate su netfilter.

\end{description}
%%%%%%%%%%%%%%%%%%%%%%%%%%%%%%%%%%%%%%%%%non numera l'ultima pagina sinistra

\clearpage{\pagestyle{empty}\cleardoublepage}



\chapter{VsnLib}                %crea il capitolo
%%%%%%%%%%%%%%%%%%%%%%%%%%%%%%%%%%%%%%%%%imposta l'intestazione di pagina
\lhead[\fancyplain{}{\bfseries\thepage}]{\fancyplain{}{\bfseries\rightmark}}
Strato di compatibilit\`a tra pacchetti netlink e stack di rete virtuali.

\section{Il Progetto}                 %crea la sezione
Gli stack analizzati in precedenza offrono a grandi linee la stessa tipologia di servizio seppur ognuna con le proprie caratteristice.\\
Nessuno dei progetti ha per\`o tenuto in considerazione l'idea di utilizzare un'interfaccia di configurazione che fosse standard e pertanto \`e necessario usare le funzioni specifiche per ognuno di questi progetti, costringendo il programmatore a cambiare modalit\`a di interazione da stack a stack.\\
Il progetto di creare questa libreria nasce proprio da questa esigenza, ovvero cercare di uniformare le interfacce di comunicazione in modo tale che l'utente non sia costretto ad adattarsi ogni qual volta decida di cambiare stack.\\
\section{VnsLib}
\subsection{Preambolo}
Per configurare uno stack di rete programmi come ip generano un pacchetto netlink con le informazioni necessarie lo inviano al kernel che lo elabora esegue le operazioni e genera un messaggio di errore nel quale esiste un flag contenente il numero di errore generato (0 in caso di successo), a questo messaggio \`e associato un payload di risposta che pu\`o essere anche un feedback per l'utente che sta interagendo con il programma.\\

\subsection{Ambiente}
La necessit\`a primaria era quella di catturare questi netlink e per farlo la strada era quella di intercettare la system call di invio della richiesta contenente appunto il payload, questo momento \`e quello in cui viene eseguita la sendto.\\
In questo punto interviene la nostra libreria, che si interpone e di fatto fa le veci del kernel, al quale il pacchetto netlink non arriver\`a mai realmente.\\
Il progetto inizialmente utilizzava come motore di cattura delle system call la libreria purelibc\footnote{http://wiki.virtualsquare.org/wiki/index.php/PureLibc}, in seguito per\`o si \`e giunti alla conclusione che costruire un modulo ad hoc all'interno del sistema vuos fosse un'alternativa pi\`u versatile, immediata e pulita, inoltre vuos ha un sistema di debug built in che rende pi\`u semplice l'individuazione dei problemi legati allo sviluppo.\\
Il modulo in questione si chiama unrealvsnlib per analogia alla libreria ma ognuno potrebbe costruire un suo modulo a seconda di quello che intende controllare.\\

\subsection{Core}
La libreria risulta come uno strato di compatibilit\`a tra netlink e le specifiche configurazioni degli stack.\\
Vediamo come \`e composta:
\begin{description}                     %crea un elenco descrittivo
  \item[VnsLib:] Il primo stato si occupa di inizializzare la libreria in base allo stack che si intende utilizzare, in questo modo possiamo caricare dinamicamente solo il modulo che ci interessa.
  \item[Modules:] Sono la parte specifica della libreria, essi contengono l'intestazione delle funzioni generiche che si occupano di chiamare quelle particolari per ogni stack.
\end{description}
\begin{figure}[h]                       %crea l'ambiente figura; [h] sta
                                        %   per here, cio� la figura va qui
\begin{center}                          %centra nel mezzo della pagina
                                        %   la figura
\includegraphics[width=15cm]{vsnlib_scheme}%inserisce una figura larga 5cm
                                        %se si vuole usare va scommentata
%
%%%%%%%%%%%%%%%%%%%%%%%%%%%%%%%%%%%%%%%%%inserisce la legenda ed etichetta
                                        %   la figura con \label{fig:prima}
\caption[mappa concettuale libreria]{mappa concettuale libreria}\label{fig:map}
\end{center}
\end{figure}



\subsection{Moduli}
La potenza dei moduli risiede nel fatto che chiunque abbia intenzione di costruire uno stack personalizzato, o di utilizzarne uno per il quale non esiste un modulo, pu\`o facilmente realizzare il proprio e la libreria si occuper\`a di farne il caricamento qualora fosse richiesto.
\section{Sviluppi Futuri}
%%%%%%%%%%%%%%%%%%%%%%%%%%%%%%%%%%%%%%%%%non numera l'ultima pagina sinistra
\clearpage{\pagestyle{empty}\cleardoublepage} 



\chapter{Casi d'uso}                %crea il capitolo
%%%%%%%%%%%%%%%%%%%%%%%%%%%%%%%%%%%%%%%%%imposta l'intestazione di pagina
\lhead[\fancyplain{}{\bfseries\thepage}]{\fancyplain{}{\bfseries\rightmark}}
Di seguito alcuni esempi di utilizzo della libreria.
Le dimostrazioni seguenti sono state effettuate su una macchina con architettura amd64 e con installato il sistema operativo debian in versione sid (quindi unstable) ma sono state testate e riprodotteanche su altre configurazioni e distribuzioni GNU/Linux.

\section{Esempi}                 %crea la sezione

%%%%%%%%%%%%%%%%%%%%%%%%%%%%%%%%%%%%%%%%%non numera l'ultima pagina sinistra
\clearpage{\pagestyle{empty}\cleardoublepage} 



%%%%%%%%%%%%%%%%%%%%%%%%%%%%%%%%%%%%%%%%%per fare le conclusioni
\chapter*{Conclusioni}
%%%%%%%%%%%%%%%%%%%%%%%%%%%%%%%%%%%%%%%%%imposta l'intestazione di pagina
\rhead[\fancyplain{}{\bfseries
Conclusioni}]{\fancyplain{}{\bfseries\thepage}}
\lhead[\fancyplain{}{\bfseries\thepage}]{\fancyplain{}{\bfseries
CONCLUSIONI}}
%%%%%%%%%%%%%%%%%%%%%%%%%%%%%%%%%%%%%%%%%aggiunge la voce Conclusioni
                                        %   nell'indice
\addcontentsline{toc}{chapter}{Conclusioni}
Lo state dell'arte al momento in cui si scrive \`e una libreria in grado di configurare uno stack per LWIPv6 completamente, mentre gli altri moduli sono in fase di sviluppo ma esistono gi\`a dei proof of concept (moduli in fase sperimentale) all'interno del repo su github\footnote{http://github.com/simonepreite/vnslib}.
I risultati dei test effettuati, nonostante il progetto sia tutt'ora in fase di sviluppo, sono significativi: si riesce a configurare completamente uno stack specifico in maniera generica ed usando le interfacce proposte dal kernel Linux.\\
L'utilizzo di queste tipologie di stack dipende strettamente dalla diffusione del protocollo IPv6 e dei sistemi embedded.\\
Lo sviluppo di queste tecnologie \`e punti chiave per il testing del procollo IPv6. Solo mantenendo costante la ricerca si pu\`o pensare di creare sistemi stabili quando IPv6 sar\`a
La ricerca e lo sviluppo di questi meccanismi sono necessari per ottenere un sistema performante e stabile quando IPv6 sar\`a l'ordinario. Vanno inoltre considerati i vantaggi conseguenti nelle reti virtuali che gi\`a ora vengono utilizzate per la sperimentazione e che in futuro saranno sicuramente di larga diffusione.\\
L'indipendenza dalle infrastrutture fisiche \`e un vantaggio non indifferente, significa che le reti ed i sistemi possono essere creati, modificati o ricostruiti da zero con estrema facilit\`a. Basti pensare ad una rete di macchine virtuali che, trattandosi di software, pu\`o essere ampliata con nuovi elaboratori virtuali, spenta e modificata senza dover invesitire in nuovo hardware.\\
Quelli esposti sono solo alcuni dei vantaggi sufficienti a lasciare intendere la direzione verso cui questo settore sta proseguendo.


\chapter*{Sviluppi Futuri}
%%%%%%%%%%%%%%%%%%%%%%%%%%%%%%%%%%%%%%%%%imposta l'intestazione di pagina
\rhead[\fancyplain{}{\bfseries
Sviluppi Futuri}]{\fancyplain{}{\bfseries\thepage}}
\lhead[\fancyplain{}{\bfseries\thepage}]{\fancyplain{}{\bfseries
SVILUPPI FUTURI}}
%%%%%%%%%%%%%%%%%%%%%%%%%%%%%%%%%%%%%%%%%aggiunge la voce Conclusioni
                                        %   nell'indice
\addcontentsline{toc}{chapter}{Sviluppi Futuri}
In fase di progettazione non sono stati posti limiti alla crescita del progetto e la sua modularit\`a \`e legata principalmente a questo aspetto.\\
L'augurio \`e che questo progetto continui a crescere arricchendosi di funzionalit\`a.\\
In questa fase ci si \`e occupati delle configurazioni e del routing che deve essere completata e testata prima di poter affermarne la stabilit\`a.\\
Senz'altro aumentare il supporto agli stack meno conosciuti ed a quelli che in futuro saranno sviluppati \`e il prosieguo pi\`u ovvio per VSNLib.
Tra le altre ipotesi di ampliamento c'\`e quella di gestione del filtering, l'intento in questo caso sar\`a fornire un'interfaccia identica a quella di iptables ma per settare le regole di filtraggio sugli stack virtuali.\\


%%%%%%%%%%%%%%%%%%%%%%%%%%%%%%%%%%%%%%%%%imposta l'intestazione di pagina
\renewcommand{\chaptermark}[1]{\markright{\thechapter \ #1}{}}
\lhead[\fancyplain{}{\bfseries\thepage}]{\fancyplain{}{\bfseries\rightmark}}
\appendix                               %imposta le appendici
\chapter{Core}               %crea l'appendice
\label{code:AppA}
%%%%%%%%%%%%%%%%%%%%%%%%%%%%%%%%%%%%%%%%%imposta l'intestazione di pagina
\rhead[\fancyplain{}{\bfseries \thechapter \:Core}]
{\fancyplain{}{\bfseries\thepage}}
vsnshared.h
\begin{lstlisting}[style=CscriptStyle]
#ifndef _VSNSHARED_H
#define _VSNSHARED_H

#include <asm/types.h>
#include <linux/netlink.h>
#include <linux/rtnetlink.h>
#include <arpa/inet.h>

struct response{
	int error; //0 for success
};

extern void handle_vsnlib(void* buf, size_t len, void* nif, void* stack);
extern int init_vsnlib(char* stack);

#endif

\end{lstlisting}
vsnlib.h
\begin{lstlisting}[style=CscriptStyle]
#ifndef _VSNLIB_H
#define _VSNLIB_H

#include "vsnshared.h"

#define  API_TABLE_SIZE 6

/* struct definition */
struct config{
	char* addr;
	int mask;
	int family;
	void* interface;
	void* stack; // stack file descriptor
	int operation;
};

typedef struct response* (*generic_api)(struct config* cfg);
struct gen_api{
  char* fun_name;
  generic_api real_fun;
};

static struct gen_api api_table[]={
  {"adddellink", NULL},
  {"getsetlink",NULL},
  {"adddeladdr", NULL},
  {"getaddr", NULL},
  {"adddelroute", NULL},
  {"getroute", NULL}
};

/* client side */

struct nlmsghdr* rtm_newroute_c(struct nlmsghdr* header, void* nif, void* stack);
struct nlmsghdr* rtm_delroute_c(struct nlmsghdr* header, void* nif, void* stack);
struct nlmsghdr* rtm_getroute_c(struct nlmsghdr* header, void* nif, void* stack);
struct nlmsghdr* rtm_newaddr_c(struct nlmsghdr* header, void* nif, void* stack);
struct nlmsghdr* rtm_deladdr_c(struct nlmsghdr* header, void* nif, void* stack);
struct nlmsghdr* rtm_getaddr_c(struct nlmsghdr* header, void* nif, void* stack);
struct nlmsghdr* rtm_newlink_c(struct nlmsghdr* header, void* nif, void* stack);
struct nlmsghdr* rtm_dellink_c(struct nlmsghdr* header, void* nif, void* stack);
struct nlmsghdr* rtm_getlink_c(struct nlmsghdr* header, void* nif, void* stack);
struct nlmsghdr* rtm_setlink_c(struct nlmsghdr* header, void* nif, void* stack);

typedef struct nlmsghdr* vsn_cli(struct nlmsghdr* header, void* nif, void* stack);
static vsn_cli *vsncli_fun_table[]={
	/* NEW/DEL/GET/SET link */
  rtm_newlink_c,
  rtm_dellink_c,
  rtm_getlink_c,
  rtm_setlink_c,
	/* NEW/DEL/GET addr */
  rtm_newaddr_c,
  rtm_deladdr_c,
  rtm_getaddr_c,
  NULL,
	/* NEW/DEL/GET route */
  rtm_newroute_c,
  rtm_delroute_c,
  rtm_getroute_c,
  NULL
};

#endif

\end{lstlisting}

vsnlib.c
\begin{lstlisting}[style=CscriptStyle]
#include <stdio.h>
#include <stdlib.h>
#include <fcntl.h>
#include <unistd.h>
#include <dirent.h>
#include <string.h>
#include <dlfcn.h>

#include <vsnlib.h>
#include <vsnshared.h>


struct ret_val{
  struct nlmsghdr* res;
  struct ret_val* next;
};

char* combine_stack_handler(char* fun_name, char* stack){
    //rimuove il .so
    char* combo;
    int len_stack = strlen(stack)-1;
    combo = malloc(sizeof(char)*(strlen(fun_name)+(len_stack-1)));
    strcpy(combo, stack);
    combo[len_stack-2] = '_';
    combo[len_stack-1] = '\0';
    strcat(combo, fun_name);
    return combo;
}


int init_vsnlib(char* stack){
  const char *error;
  char* f;
  void *stack_module;

  stack_module = dlopen(stack,RTLD_LAZY);
  if(!stack_module){
    fprintf(stderr, "Cannot open stack library: %s\n",
    dlerror());
    return -1;
  }
  for(int i=0; i < API_TABLE_SIZE; i++){
    api_table[i].real_fun = dlsym(stack_module, api_table[i].fun_name);
    if ((error = dlerror())) {
      fprintf(stderr, "Couldn't find %s: %s\n", api_table[i].fun_name, error);
      exit(1);
    }
  }
  return 0; //if success
}

void handle_vsnlib(void* buf, size_t len, void* nif, void* stack){
  struct nlmsghdr* header = (struct nlmsghdr*) buf;
  struct nlmsghdr* ret_pkg;
  while (NLMSG_OK(header, len)) {
    /*int err;*/
    int type;

    if (header->nlmsg_type == NLMSG_DONE)
      break;
    type=header->nlmsg_type - RTM_BASE;
    header->nlmsg_pid=getpid();
    if (type >= 0 && vsncli_fun_table[type] != NULL) {
        struct ret_val* app = malloc(sizeof(struct ret_val));
        (vsncli_fun_table[type](header, nif, stack));
    }
    header = NLMSG_NEXT(header, len);
  }
}

struct nlmsghdr* rtm_newlink_c(struct nlmsghdr* header, void* nif, void* stack){
  struct ifinfomsg* ifi = (struct ifinfomsg*)(NLMSG_DATA(header));
  struct config generic_conf;
  struct response* generic_resp;
  generic_conf.interface = nif;
  generic_conf.operation = ifi->ifi_flags;
  generic_resp = api_table[0].real_fun(&generic_conf);
}

struct nlmsghdr* rtm_dellink_c(struct nlmsghdr* header, void* nif, void* stack){
  struct ifinfomsg* ifi = (struct ifinfomsg*)(NLMSG_DATA(header));
  struct config* generic_conf;
  struct response* generic_resp;
  generic_resp = api_table[0].real_fun(generic_conf);
}

struct nlmsghdr* rtm_getlink_c(struct nlmsghdr* header, void* nif, void* stack){
  struct ifinfomsg* ifi = (struct ifinfomsg*)(NLMSG_DATA(header));
  struct config* generic_conf;
  struct response* generic_resp;
  generic_resp = api_table[1].real_fun(generic_conf);
}

struct nlmsghdr* rtm_setlink_c(struct nlmsghdr* header, void* nif, void* stack){
  struct ifinfomsg* ifi = (struct ifinfomsg*)(NLMSG_DATA(header));
  struct config* generic_conf;
  struct response* generic_resp;
  generic_resp = api_table[1].real_fun(generic_conf);
}

struct nlmsghdr* rtm_newaddr_c(struct nlmsghdr* header, void* nif, void* stack){
  struct ifaddrmsg* ifa = (struct ifaddrmsg*)(NLMSG_DATA(header));
  struct config generic_conf;
  struct response* generic_resp;
  char str6[INET6_ADDRSTRLEN];
  char* str4;
  generic_conf.operation=header->nlmsg_type;
  generic_conf.interface=nif;
  generic_conf.mask=ifa->ifa_prefixlen;
  generic_conf.family=ifa->ifa_family;
  struct rtattr* attr = (struct rtattr*)((void*)(((char*)ifa)+ sizeof(struct ifaddrmsg)));
  if(ifa->ifa_family == AF_INET6){
    struct in6_addr* ip6_a = (struct in6_addr*)RTA_DATA(attr);
    inet_ntop(ifa->ifa_family, ip6_a, str6, INET6_ADDRSTRLEN);
    inet_pton(ifa->ifa_family, str6, ip6_a);
    generic_conf.addr=str6;
  }
  else{
    struct in_addr ip4_a = *((struct in_addr*)RTA_DATA(attr));
    str4 = inet_ntoa(ip4_a);
    generic_conf.addr=str4;
  }
  generic_conf.stack=stack;
  generic_resp = api_table[2].real_fun(&generic_conf);
}

struct nlmsghdr* rtm_deladdr_c(struct nlmsghdr* header, void* nif, void* stack){
  rtm_newaddr_c(header, nif, stack);
}

struct nlmsghdr* rtm_getaddr_c(struct nlmsghdr* header, void* nif, void* stack){
  struct ifaddrmsg* ifa = (struct ifaddrmsg*)(NLMSG_DATA(header));
  struct config* generic_conf;
  struct response* generic_resp;
  generic_resp = api_table[3].real_fun(generic_conf);
}

struct nlmsghdr* rtm_newroute_c(struct nlmsghdr* header, void* nif, void* stack){
  struct rtmsg* rtm = (struct rtmsg*)(NLMSG_DATA(header));
  struct config generic_conf;
  struct response* generic_resp;
  char str6[INET6_ADDRSTRLEN];
  char* str4;

  generic_conf.operation=header->nlmsg_type;
  generic_conf.interface=nif;
  generic_conf.family=rtm->rtm_family;
  generic_conf.stack=stack;
  struct rtattr* attr = (struct rtattr*)((void*)(((char*)rtm)+ sizeof(struct rtmsg)));
  if(rtm->rtm_family == AF_INET6){
    struct in6_addr* ip6_a = (struct in6_addr*)RTA_DATA(attr);
    inet_ntop(rtm->rtm_family, ip6_a, str6, INET6_ADDRSTRLEN);
    generic_conf.addr=str6;
  else{
    struct in_addr ip4_a = *((struct in_addr*)RTA_DATA(attr));
    str4 = inet_ntoa(ip4_a);
    generic_conf.addr=str4;
  }
  generic_resp = api_table[4].real_fun(&generic_conf);
  }
}

struct nlmsghdr* rtm_delroute_c(struct nlmsghdr* header, void* nif, void* stack){
  rtm_newroute_c(header, nif, stack);
}

struct nlmsghdr* rtm_getroute_c(struct nlmsghdr* header, void* nif, void* stack){
  struct rtmsg* rtm = (struct rtmsg*)(NLMSG_DATA(header));
  struct config* generic_conf;
  struct response* generic_resp;
  generic_resp = api_table[5].real_fun(generic_conf);
}

\end{lstlisting}

\chapter{Modules}               %crea l'appendice
\label{code:AppB}
%%%%%%%%%%%%%%%%%%%%%%%%%%%%%%%%%%%%%%%%%imposta l'intestazione di pagina
\rhead[\fancyplain{}{\bfseries \thechapter \:Modules}]
{\fancyplain{}{\bfseries\thepage}}
vsnmodules.h
\begin{lstlisting}[style=CscriptStyle]

\end{lstlisting}

LWIPv6.c
\begin{lstlisting}[style=CscriptStyle]
#include <stdio.h>
#include <vsnlib.h>
#include <lwipv6.h>


struct response* adddeladdr(struct config* cfg){
  struct ip_addr addr;
  struct ip_addr mask;
  uint16_t* p_mask = NULL;

  if(cfg->family == AF_INET6){
    uint16_t mask_app[8]={0,0,0,0,0,0,0,0};
    struct in6_addr ip6;

    uint16_t* p = (uint16_t*)(&ip6);
    p_mask = (uint16_t*)(&mask_app);

    for(int i=0; i<8; i++){
      *p=0;
      p++;
    }
    p = (uint16_t*)(&ip6);
    inet_pton(cfg->family, cfg->addr, &ip6);

    while(cfg->mask > 0){
      *p_mask = 0xffff;
      cfg->mask-= 16;
      p_mask++;
    }

    p_mask = (uint16_t*)(&mask_app);

    printf("ipv6 addr: %04X::%04X::%04X::%04X::%04X::%04X::%04X::%04X\n",htons(*(p)),htons(*(p+1)),htons(*(p+2)),htons(*(p+3)),htons(*(p+4)),htons(*(p+5)),htons(*(p+6))htons(*(p+7)));
    IP6_ADDR(&addr,htons(*(p)),htons(*(p+1)),htons(*(p+2)),htons(*(p+3)),htons(*(p+4)),htons(*(p+5)),htons(*(p+6)),htons(*(p+7)));
    printf("mask: %04X::%04X::%04X::%04X::%04X::%04X::%04X::%04X\n",*(p_mask),*(p_mask+1),*(p_mask+2),*(p_mask+3),*(p_mask+4),*(p_mask+5),*(p_mask+6),*(p_mask+7));
    IP6_ADDR(&mask,*(p_mask),*(p_mask+1),*(p_mask+2),*(p_mask+3),*(p_mask+4),*(p_mask+5),*(p_mask+6),*(p_mask+7));
  }
  else{
    struct in_addr ip4;
    uint16_t lwip_mask4[4] = {0,0,0,0};
    p_mask = (uint16_t*)(&lwip_mask4);
    while(cfg->mask > 0){
      *p_mask=255;
      cfg->mask -= 8;
      p_mask++;
    }
    inet_aton(cfg->addr, &ip4);
    unsigned char ip4_dot[4];
    int j=0;
    for(int i=0; i<4; i++){
      ip4_dot[i]=(ip4.s_addr >> j) & 0xFF;
      j=j+8;
    }
    printf("ipv4 address: %d.%d.%d.%d\n",ip4_dot[0],ip4_dot[1],ip4_dot[2],ip4_dot[3]);
    IP64_ADDR(&addr,ip4_dot[0],ip4_dot[1],ip4_dot[2],ip4_dot[3]);
    printf("ipv4 mask: %d.%d.%d.%d\n",lwip_mask4[0],lwip_mask4[1],lwip_mask4[2],lwip_mask4[3]);
    IP64_MASKADDR(&mask,lwip_mask4[0],lwip_mask4[1],lwip_mask4[2],lwip_mask4[3]);
  }
  if(cfg->operation == RTM_NEWADDR){
    printf("error add addr lwipv6: %d\n", lwip_add_addr((struct netif*)(cfg->interface),&addr,&mask));
  }
  else if (cfg->operation == RTM_DELADDR) {
    printf("error del addr lwipv6: %d\n", lwip_del_addr((struct netif*)(cfg->interface),&addr,&mask));
  }
}

struct response* getaddr(struct config* cfg){

}

struct response* adddellink(struct config* cfg){
  printf("adddellink\n");
  if(cfg->operation == 1)
    printf("ifup: %d\n", lwip_ifup((struct netif*)(cfg->interface)));
  else
    printf("ifdown: %d\n", lwip_ifdown((struct netif*)(cfg->interface)));
}

struct response* getsetlink(struct config* cfg){

}

struct response* adddelroute(struct config* cfg){
  struct ip_addr addr;

  if(cfg->family==AF_INET6){
    struct in6_addr address;
    uint16_t* p = (uint16_t*)(&address);
    p = (uint16_t*)(&address);
    inet_pton(cfg->family, cfg->addr, &address);
    printf("ipv6 addr: %04X::%04X::%04X::%04X::%04X::%04X::%04X::%04X\n",htons(*(p)),htons(*(p+1)),htons(*(p+2)),htons(*(p+3)),htons(*(p+4)),htons(*(p+5)),htons(*(p+6))htons(*(p+7)));
    IP6_ADDR(&addr,htons(*(p)),htons(*(p+1)),htons(*(p+2)),htons(*(p+3)),htons(*(p+4)),htons(*(p+5)),htons(*(p+6)),htons(*(p+7)));
  }
  else{
    struct in_addr ip4;
    inet_aton(cfg->addr, &ip4);
    unsigned char ip4_dot[4];
    int j=0;
    for(int i=0; i<4; i++){
      ip4_dot[i]=(ip4.s_addr >> j) & 0xFF;
      j=j+8;
    }
    printf("ipv4 route: %d.%d.%d.%d\n",ip4_dot[0],ip4_dot[1],ip4_dot[2],ip4_dot[3] );
    IP64_ADDR(&addr,ip4_dot[0],ip4_dot[1],ip4_dot[2],ip4_dot[3]);
  }

  if(cfg->operation==RTM_NEWROUTE){
    printf("error add route lwipv6 %d\n", lwip_add_route((struct stack*)cfg->stack, IP_ADDR_ANY, IP_ADDR_ANY, &addr, (struct netif*)(cfg->interface),0));
  }
  else if(cfg->operation==RTM_DELROUTE){
    printf("error del route lwipv6: %d\n", lwip_del_route((struct stack*)cfg->stack, IP_ADDR_ANY, IP_ADDR_ANY, &addr, (structnetif*)(cfg->interface), 0));
  }
}

struct response* getroute(struct config* cfg){

}

\end{lstlisting}

\chapter{Test}             %crea l'appendice
\label{code:AppC}
\rhead[\fancyplain{}{\bfseries \thechapter \:Test}]
{\fancyplain{}{\bfseries\thepage}}

minlibnetlink.h
\begin{lstlisting}[style=CscriptStyle]
#ifndef _MINLIBNETLINK_H
#define _MINLIBNETLINK_H
#include <stdio.h>
#include <fcntl.h>
#include <stdlib.h>
#include <unistd.h>
#include <string.h>
#include <lwipv6.h>
#include <sys/socket.h>
#include <sys/select.h>
#include <netinet/in.h>

#include <asm/types.h>
#include <linux/netlink.h>
#include <linux/rtnetlink.h>

#include <vsnshared.h>

struct info{
  struct rtattr attr;
  unsigned char ip6_address[sizeof(struct in6_addr)];
};

struct addr_payload{
  struct nlmsghdr header;
  struct ifaddrmsg ifa;
  struct info attribute[2];
};

struct route_payload{
  struct nlmsghdr header;
  struct rtmsg rtm;
  struct info attribute;
};

struct link_payload{
  struct nlmsghdr header;
  struct ifinfomsg ifi;
};

void fill_buf_link(struct link_payload* buffer, int mode);
void fill_buf_route(struct route_payload* buffer, char* IPv6, int mode, int modeIP);
void fill_buf_addr(struct addr_payload* buffer, char* IP, int mask, int mode, int modeIP);

#endif
\end{lstlisting}

minlibnetlink.c
\begin{lstlisting}[style=CscriptStyle]
#include "minlibnetlink.h"

void fill_buf_link(struct link_payload* buffer, int mode){

  // nlmsghdr init
  buffer->header.nlmsg_len=32;
  buffer->header.nlmsg_type=RTM_NEWLINK;
  buffer->header.nlmsg_flags=NLM_F_REQUEST|NLM_F_ACK;

  buffer->ifi.ifi_flags=mode;
  buffer->header.nlmsg_seq=102;
  buffer->header.nlmsg_pid=0;
  // ifaddr init
  buffer->ifi.ifi_family=AF_UNSPEC;
  buffer->ifi.ifi_type=AF_NETROM; //maschera
  buffer->ifi.ifi_index=0;
  buffer->ifi.ifi_change=0x1;
}

void fill_buf_route(struct route_payload* buffer, char* IPv6, int mode, int modeIP){
  inet_pton(modeIP, IPv6, buffer->attribute.ip6_address);

  // rtattr init
  buffer->attribute.attr.rta_len=20;
  buffer->attribute.attr.rta_type=RTA_GATEWAY;
  // nlmsghdr init
  buffer->header.nlmsg_len=64;
  buffer->header.nlmsg_type=mode;

  buffer->header.nlmsg_seq=101;
  buffer->header.nlmsg_pid=0;
  // rtmsg init
  if(modeIP==AF_INET6){
      buffer->rtm.rtm_family=AF_INET6;
  }
  else{
    buffer->rtm.rtm_family=AF_INET;
  }
  buffer->rtm.rtm_dst_len=0; //maschera
  buffer->rtm.rtm_src_len=0;
  buffer->rtm.rtm_tos=0;
  buffer->rtm.rtm_table=RT_TABLE_MAIN;
  buffer->rtm.rtm_protocol=RTPROT_UNSPEC;
  buffer->rtm.rtm_scope=RT_SCOPE_NOWHERE;
  buffer->rtm.rtm_type=RTN_UNSPEC;
  buffer->rtm.rtm_flags=0;
  if(mode==RTM_NEWROUTE){
    buffer->header.nlmsg_flags=NLM_F_REQUEST|NLM_F_ACK|NLM_F_EXCL|NLM_F_CREATE;
  }
  else if(mode==RTM_DELROUTE){
    buffer->header.nlmsg_flags=NLM_F_REQUEST|NLM_F_ACK;
  }
}

// generea lo stesso pacchetto netlink generato da ip addr add <IPv6/mask> dev <device_name>

void fill_buf_addr(struct addr_payload* buffer, char* IP, int mask, int mode, int modeIP){
  inet_pton(modeIP, IP, buffer->attribute[0].ip6_address);
  inet_pton(modeIP, IP, buffer->attribute[1].ip6_address);
  if(modeIP == AF_INET6){
    // rtattr init
    buffer->attribute[0].attr.rta_len=20;
    buffer->attribute[1].attr.rta_len=20;
    buffer->header.nlmsg_len=64;
  }
  else{
    buffer->attribute[0].attr.rta_len=8;
    buffer->attribute[1].attr.rta_len=8;
    buffer->header.nlmsg_len=40;
  }
  buffer->attribute[0].attr.rta_type=IFA_LOCAL;
  buffer->attribute[1].attr.rta_type=IFA_ADDRESS;
  buffer->ifa.ifa_family=modeIP;
  buffer->header.nlmsg_seq=100;
  buffer->header.nlmsg_pid=0;
  // ifaddr init
  buffer->ifa.ifa_prefixlen=mask; //maschera
  buffer->ifa.ifa_flags=0;
  buffer->ifa.ifa_scope=RT_SCOPE_UNIVERSE;
  buffer->ifa.ifa_index = 0;
  buffer->header.nlmsg_type=mode;
  if(mode==RTM_NEWADDR){
    buffer->header.nlmsg_flags=NLM_F_REQUEST|NLM_F_ACK|NLM_F_EXCL|NLM_F_CREATE;
  }
  else{
    buffer->header.nlmsg_flags=NLM_F_REQUEST|NLM_F_ACK;
  }
}
\end{lstlisting}

test4\_vsnlib.c
\begin{lstlisting}[style=CscriptStyle]
#include "minlibnetlink.h"

#define BUFSIZE 1024
char buf[BUFSIZE];

int main(int argc, char** argv){

  struct sockaddr_in serv_addr;
  int fd;

  struct stack *stack;
  struct netif *nif;
  void* buf;
  void* buf_r;
  void* buf_l;
  struct response* ret;
  struct ip_addr addr_6;

  struct addr_payload buf_addr;
  struct route_payload buf_route;
  struct link_payload buf_link;

  struct ip_addr addr, mask;

  buf=&buf_addr.header;
  buf_r=&buf_route.header;

  // call vsnlib
  if(init_vsnlib(argv[1])==-1){
    perror("Cannot init vsnlib");
  }

  #ifdef LWIPV6DL
  /* Run-time load the library (if requested) */
  if ((handle=loadlwipv6dl()) == NULL) {
    perror("LWIP lib not loaded");
    exit(-1);
  }
  #endif
  /* define a new stack */
  if((stack=lwip_stack_new())==NULL){
    perror("Lwipstack not created");
    exit(-1);
  }

  /* add an interface */
  if((nif=lwip_vdeif_add(stack,argv[2]))==NULL){
    perror("Interface not loaded");
    exit(-1);
  }

  fill_buf_link(&buf_link, 1);
  handle_vsnlib(&buf_link, buf_link.header.nlmsg_len, (void*)nif, (void*)stack);

  fill_buf_addr(&buf_addr, "192.168.250.20", 24, RTM_NEWADDR, AF_INET);
  handle_vsnlib(buf, buf_addr.header.nlmsg_len, (void*)nif, (void*)stack);


  fill_buf_route(&buf_route, "192.168.250.2", RTM_NEWROUTE, AF_INET);
  handle_vsnlib(buf_r, buf_route.header.nlmsg_len, (void*)nif, (void*)stack);

  memset((char *) &serv_addr,0,sizeof(serv_addr));
  serv_addr.sin_family      = AF_INET;
  serv_addr.sin_addr.s_addr = inet_addr("192.168.250.1");
  serv_addr.sin_port        = htons(atoi("9999"));

    /* create a TCP lwipv6 socket */
    if((fd=lwip_msocket(stack,PF_INET,SOCK_STREAM,0))<0) {
      perror("Socket opening error");
      exit(-1);
    }
    /* connect it to the address specified as argv[1] port argv[2] */
    if (lwip_connect(fd,(struct sockaddr *)(&serv_addr),sizeof(serv_addr)) < 0) {
      perror("Socket connecting error");
      exit(-1);
    }

    while(1) {
      fd_set rfds;
      int n;
      FD_ZERO(&rfds);
      FD_SET(STDIN_FILENO,&rfds);
      FD_SET(fd,&rfds);
      /* wait for input both from stdin and from the socket */
      lwip_select(fd+1,&rfds,NULL,NULL,NULL);
      /* copy data from the socket to stdout */
      if(FD_ISSET(fd,&rfds)) {
        if((n=lwip_read(fd,buf,BUFSIZE)) == 0)
          exit(0);
        write(STDOUT_FILENO,buf,n);
      }
      /* copy data from stdin to the socket */
      if(FD_ISSET(STDIN_FILENO,&rfds)) {
        if((n=read(STDIN_FILENO,buf,BUFSIZE)) == 0)
          exit(0);
        lwip_write(fd,buf,n);
      }
    }
    fill_buf_route(&buf_route, "192.168.250.1", RTM_DELROUTE, AF_INET);
    handle_vsnlib(buf_r, buf_route.header.nlmsg_len, (void*)nif, (void*)stack);

    fill_buf_addr(&buf_addr, "192.168.250.20", 64, RTM_DELADDR, AF_INET);
    handle_vsnlib(buf, buf_addr.header.nlmsg_len, (void*)nif, (void*)stack);

    fill_buf_link(&buf_link, 0);
    handle_vsnlib(&buf_link, buf_link.header.nlmsg_len, (void*)nif, (void*)stack);
}
\end{lstlisting}

test6\_vsnlib.c
\begin{lstlisting}[style=CscriptStyle]
#include "minlibnetlink.h"

#define BUFSIZE 1024
char buf[BUFSIZE];

int main(int argc, char** argv){

  struct sockaddr_in6 serv_addr;
  int fd;

  struct stack *stack;
  struct netif *nif;
  void* buf;
  void* buf_r;
  void* buf_l;
  struct response* ret;
  struct ip_addr addr_6;

  struct addr_payload buf_addr;
  struct route_payload buf_route;
  struct link_payload buf_link;

  buf=&buf_addr.header;
  buf_r=&buf_route.header;

  // call vsnlib
  if(init_vsnlib(argv[1])==-1){
    perror("Cannot init vsnlib");
  }

  #ifdef LWIPV6DL
  /* Run-time load the library (if requested) */
  if ((handle=loadlwipv6dl()) == NULL) {
    perror("LWIP lib not loaded");
    exit(-1);
  }
  #endif
  /* define a new stack */
  if((stack=lwip_stack_new())==NULL){
    perror("Lwipstack not created");
    exit(-1);
  }

  /* add an interface */
  if((nif=lwip_vdeif_add(stack,argv[2]))==NULL){
    perror("Interface not loaded");
    exit(-1);
  }

  fill_buf_link(&buf_link, 1);
  handle_vsnlib(&buf_link, buf_link.header.nlmsg_len, (void*)nif, (void*)stack);

  fill_buf_addr(&buf_addr, "2001:db8:0:f101::3", 64, RTM_NEWADDR, AF_INET6);
  handle_vsnlib(buf, buf_addr.header.nlmsg_len, (void*)nif, (void*)stack);

  fill_buf_route(&buf_route, "2001:db8:0:f101::2", RTM_NEWROUTE, AF_INET6);
  handle_vsnlib(buf_r, buf_route.header.nlmsg_len, (void*)nif, (void*)stack);

  memset((char *) &serv_addr,0,sizeof(serv_addr));
    serv_addr.sin6_family = PF_INET6;
    char str[INET6_ADDRSTRLEN];
    int reto = inet_pton(PF_INET6, "2001:db8:0:f101::1", serv_addr.sin6_addr.s6_addr);
    inet_ntop(PF_INET6, serv_addr.sin6_addr.s6_addr, str, INET6_ADDRSTRLEN);
    serv_addr.sin6_port = htons(atoi("9999"));

    /* create a TCP lwipv6 socket */
    if((fd=lwip_msocket(stack,PF_INET6,SOCK_STREAM,0))<0) {
      perror("Socket opening error");
      exit(-1);
    }
    /* connect it to the address specified as argv[1] port argv[2] */
    if (lwip_connect(fd,(struct sockaddr *)(&serv_addr),sizeof(serv_addr)) < 0) {
      perror("Socket connecting error");
      exit(-1);
    }

    while(1) {
      fd_set rfds;
      int n;
      FD_ZERO(&rfds);
      FD_SET(STDIN_FILENO,&rfds);
      FD_SET(fd,&rfds);
      /* wait for input both from stdin and from the socket */
      lwip_select(fd+1,&rfds,NULL,NULL,NULL);
      /* copy data from the socket to stdout */
      if(FD_ISSET(fd,&rfds)) {
        if((n=lwip_read(fd,buf,BUFSIZE)) == 0)
          exit(0);
        write(STDOUT_FILENO,buf,n);
      }
      /* copy data from stdin to the socket */
      if(FD_ISSET(STDIN_FILENO,&rfds)) {
        if((n=read(STDIN_FILENO,buf,BUFSIZE)) == 0)
          exit(0);
        lwip_write(fd,buf,n);
      }
    }

    fill_buf_route(&buf_route, "2001:db8:0:f101::2", RTM_DELROUTE, AF_INET6);
    handle_vsnlib(buf_r, buf_route.header.nlmsg_len, (void*)nif, (void*)stack);

    fill_buf_addr(&buf_addr, "2001:db8:0:f101::3", 64, RTM_DELADDR, AF_INET6);
    handle_vsnlib(buf, buf_addr.header.nlmsg_len, (void*)nif, (void*)stack);

    fill_buf_link(&buf_link, 0);
    handle_vsnlib(&buf_link, buf_link.header.nlmsg_len, (void*)nif, (void*)stack);

}
\end{lstlisting}

Makefile
\begin{lstlisting}[language=bashn, basicstyle=\ttfamily\scriptsize]
CC=cc
FLAG_CC=-ldl -lpthread -llwipv6 -lvsn -g
BUILDDIR=./build/
OBJ=[$(]BUILDDIR[)]minlibnetlink.o
FLAG_PREP=-c -o

all: test4 test6

test4: buildlib [$(]OBJ[)]
  [$(]CC[)] [$(]FLAG_PREP[)] [$(]BUILDDIR[)]test4.o test4_vsnlib.c
  [$(]CC[)] -o [$(]BUILDDIR[)]test4.exe [$(]BUILDDIR[)]test4.o [$(]OBJ[)] [$(]FLAG_CC[)]

test6: buildlib [$(]OBJ[)]
  [$(]CC[)] [$(]FLAG_PREP[)] [$(]BUILDDIR)test6.o test6_vsnlib.c
  [$(]CC[)] -o [$(]BUILDDIR[)]test6.exe [$(]BUILDDIR[)]test6.o [$(]OBJ[)] [$(]FLAG_CC[)]

buildlib:
  mkdir -p [$(]BUILDDIR[)]
  [$(]CC[)] [$(]FLAG_PREP[)] [$(]BUILDDIR[)]minlibnetlink.o minlibnetlink.c

clean:
  rm -rf build
\end{lstlisting}



\begin{thebibliography}{90}             %crea l'ambiente bibliografia
\rhead[\fancyplain{}{\bfseries \leftmark}]{\fancyplain{}{\bfseries
\thepage}}
%%%%%%%%%%%%%%%%%%%%%%%%%%%%%%%%%%%%%%%%%aggiunge la voce Bibliografia
                                        %   nell'indice
\addcontentsline{toc}{chapter}{Bibliografia}
%%%%%%%%%%%%%%%%%%%%%%%%%%%%%%%%%%%%%%%%%provare anche questo comando:
%%%%%%%%%%%\addcontentsline{toc}{chapter}{\numberline{}{Bibliografia}}
\bibitem{K1} Renzo Davoli. Internet of Threads. \url{http://www.cs.unibo.it/~renzo/papers/2013.iciw.pdf}, 2013.
\bibitem{K2} Renzo Davoli. Internet of Threads: Processes as Internet Nodes. \url{http://www.cs.unibo.it/~renzo/papers/2014.IntTechIoTh.pdf}, 2014.
\bibitem{K3} Renzo Davoli. Internet of Threads. \url{http://www.cs.unibo.it/~renzo/papers/ConfGARR11_SelectedPapers_Davoli.pdf}.
\bibitem{K4} Altran. picoTCP. \url{https://github.com/tass-belgium/picotcp}.
\bibitem{K5} Virtualsquare Team. vuos. \url{https://github.com/virtualsquare/vuos}.
\bibitem{K6} Virtualsquare Team. LWIPv6. \url{http://wiki.v2.cs.unibo.it/wiki/index.php/LWIPV6}.
\bibitem{K7} Virtualsquare Team. UMview. \url{http://wiki.v2.cs.unibo.it/wiki/index.php/UMview}.
\bibitem{K8} Virtualsquare Team. purelibc. \url{http://wiki.virtualsquare.org/wiki/index.php/PureLibc}.
\bibitem{K9} NetLink associazione LUGMan (Linux Users Group Mantova). \url{http://lugman.org/images/4/43/NetLink.pdf}, 2009.
\bibitem{K10} Thomas Graf. \url{https://www.infradead.org/~RFC 4291RFC 4291tgr/libnl/}.
\bibitem{K11} Thomas Graf. \url{https://github.com/tgraf/libnl}.
\bibitem{K12} Thomas Haller. \url{https://github.com/thom311/libnl}.
\bibitem{K13} LWIP. \url{https://savannah.nongnu.org/projects/lwip/}.
\bibitem{K14} Reti mesh picoTCP. \url{http://www.picotcp.com/mesh-design-guide}.
\bibitem{K15} Virtual Square. \url{http://www.virtualsquare.org}.
\bibitem{K16} Renzo Davoli. vdens. \url{https://github.com/rd235/vdens}.
\bibitem{K17} RFC 4291, \url{https://tools.ietf.org/html/rfc4291}.
\end{thebibliography}
%%%%%%%%%%%%%%%%%%%%%%%%%%%%%%%%%%%%%%%%%non numera l'ultima pagina sinistra
\clearpage{\pagestyle{empty}\cleardoublepage}



\chapter*{Ringraziamenti}
\thispagestyle{empty}
Qui possiamo ringraziare il mondo intero!!!!!!!!!!\\
Ovviamente solo se uno vuole, non \`e obbligatorio.



\end{document}
