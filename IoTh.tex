\chapter{Internet of Threads}                %crea il capitolo
%%%%%%%%%%%%%%%%%%%%%%%%%%%%%%%%%%%%%%%%%imposta l'intestazione di pagina
\lhead[\fancyplain{}{\bfseries\thepage}]{\fancyplain{}{\bfseries\rightmark}}
\pagenumbering{arabic}                  %mette i numeri arabi
\section{Paradigma}                 %crea la sezione
IoT (Internet of Things), ovvero l'internet delle cose.\\
Questo concetto vuole rapprensentare la diffusione dei sistemi embedded come veri e propri nodi di rete, possiamo definirlo il precursore del sistema che stiamo per descrivere ed \`e il sistema che ad oggi ha permesso di incontrare il nostro condizionatore o la nostra caldaia, ma addirittura il nostro tostapane, in rete.\\
Ad un certo punto si \`e sentita l'esigenza di elevare questa astrazione, nasce il concetto di IoTh (ovvero Internet of Threads).\\
L'idea diventa quella di avere processi come nodi di rete e non oggetti fisici, l'analogia \`e molto simile alla differenza tra telefoni fissi e telefoni cellulari, ovvero in passata era necessario pensare al luogo in cui una persona potesse trovarsi,, mentre attraverso l'assegnamento di un numero personale oggi possiamo comunicare direttamente con la persona desiderata\cite{K1,K2}.\\
In termini di internet questo si traduce nella possibilit\`a di migrare servizi da un capo all'altro del mondo con la semplicit\`a di un kill and start.
Una nota va anche fatta in termini di sicurezza, ogni servizio pu\`o essere eseguito con il proprio stack di rete e questo impedirebbe a software di port mapping di carpire informazioni sulla macchina che ospita detto servizio. Inoltre possedendo il proprio stack di rete possiamo eseguire i processi con un utente non privilegiato e pertanto anche un bug del demone non comprometterebbe l'intero sistema.

\section{Stack Famosi}
Diversi sono i progetti che si occupano di offrire ai processi il proprio stack di rete; tra questi ne verranno presi in esame tre, considerati pi\`u indicati per uno studio in quanto open source.\\
Ognuno di essi offre la propria implementazione di stack di rete.
\subsection{LWIP}
Nasce da Adam Dunkels ed inizialmente non forniva supporto ad IPv6.
\subsection{LWIPv6}
\subsection{PicoTCP}
\section{Netlink}
Netlink \`e un sistema IPC (Inter Process Comunication) usato perch\`e in grado di mettere in comunicazione diversi task, solitamente serve per far comunicare task in user-space con task in kernel-space ma pu\`o far interagire anche processi entrambi in user-space.\\
Studiato per essere il successore di ioctl per le configurazioni ed il monitoraggio, si propone di essere pi\`u flessibile
\subsection{Inter Process Comunication}
Molti processi necessitano di scambiare informazioni per i motivi pi\`u disparati, dalla comunicazion e di rete a quella interna ad una sola macchina.\\
Viene da se pensare che costruire programmi che non interagiscono con il mondo esterno o con altri programmi sarebbe una risorsa limitata.\\
Con IPC intendiamo quindi l'insieme delle tecnologie adottate per permettere ai processi di comunicare tra essi, che siano ospitati sulla stessa macchina o distribuiti sulla rete, tra queste tecnologie esiste appunto quella utilizzata all'interno della libreria oggetto di questo elaborato.
\subsection{Libnl}
Netlink Protocol Library Suite (libnl), \`e una raccolta di librerie ed utility che forniscono API di comunicazione netlink basate su quelle del kernel linux e comprende \cite{K10}:
\begin{description}                     %crea un elenco descrittivo
  \item[libnl] Core della libreria, offre uno strato di unificazione delle interfacce sulle quali poi si basano le altre librerie che pertanto di pendo da questa;
  \item[libnl-route] Questa libreria si occupa di fornire API per la configurazione degli elemnti della famiglia NETLINK\_ROUTE;
  \item[libnl-genl] genl significa generic netlink e questa libreria offre una versione estesa del protocollo netlink;
  \item[libnl-nf] API per configurazioni netlink basate su netfilter.

\end{description}
%%%%%%%%%%%%%%%%%%%%%%%%%%%%%%%%%%%%%%%%%non numera l'ultima pagina sinistra

\clearpage{\pagestyle{empty}\cleardoublepage} 
